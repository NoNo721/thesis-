\documentclass{article}
\usepackage{braket}
\usepackage{ctex}%add Chinese supports
\usepackage{upgreek}%add special greek symbols
\usepackage{graphicx}%include figure files
\usepackage{wrapfig}
\usepackage{subfigure}
\usepackage{amsmath}%add special math symbols
\usepackage{amssymb}
\usepackage{bm}%bold math
%\usepackage{xfrac}%add special fraction supports
\usepackage{color}
\usepackage[colorlinks, 
linkcolor=black,
anchorcolor=blue, citecolor=blue, urlcolor=blue]{hyperref}% add hypertext capabilities
\usepackage{geometry}
\geometry{a4paper, centering, scale=0.75}

%
%  Created by WW on 11/27/19
%  Copyright © WW. All rights reserved.
%
\usepackage[subfigure]{tocloft}
\renewcommand{\cftsecleader}{\cftdotfill{\cftdotsep}} 

\begin{document}\zihao{5}

\title{{\vspace{-80pt}\normalsize 现代物理评论}~\\\textbf{自旋波}}
\author{范克兰东克(J. Van Kranendonk),范弗莱克(J. H. Van Vleck)~\\哈佛大学,剑桥市,马塞诸塞州}
\date{卷30,第一期 \hfill 一月,1958}
\maketitle

{\vspace{-53pt}\centerline{——————————————————————————————————————————————}}
{\vspace{4pt}\centerline{——————————————————————————————————————————————}}

\vspace{0pt}
\tableofcontents

{\vspace{15pt}\centerline{——————————}\vspace{-10pt}}

%%===========================================================
%%===========================================================

\section{介绍} \label{sec:1}

如何去计算一个铁磁、反铁磁或者铁氧体材料中的磁化强度是一个十分复杂的问题,以至于总是需要采用某种近似。在居里温度以上,配分函数可以展开为交换积分与$kT$之比的幂级数,但是收敛速度非常慢,而且每个后续项的计算难度越来越大。外斯(Weiss)分子场或其衍生方法提供了一种半经验半理论的方式来解决问题,而且如果只作定性考虑,即使在低于居里温度的情况下这种方法也是有效的。外斯-贝特-皮尔斯(Weiss-Bethe-Peierls)方法可以看作是外斯分子场方法的改进,其中群集内的相互作用得到了准确处理,但其与周围环境的耦合仅仅是现象学上的。

极低温下的磁化强度与绝对零度时仅稍有差异,在这种情况下上述方法都不够让人满意。而对于由规则排列的原子组成的晶体,我们有一种自旋波的方法,这正是本论文的主题。我们的目标有两个:首先,将很多分散在不同文献中的结果以一种统一的方式整理在一处;其次,引入了量子力学理论的简化版本。我们的理论基于用一个简谐振子系统来近似磁性自旋系统,与基于产生湮灭算符的传统布洛赫(Bloch)、荷斯坦(Holstein)和普里马科夫(Primakoff)理论的差别很大程度上仅仅是语义学上的。尽管如此,有一些读者,他们觉得简谐振子相比于量子力学场论更加符合直觉,对他们而言这种方法可能更易于理解。

我们的处理方式(除了特别说明的)都是基于磁性固体的海森堡(Heisenberg)或者“局域自旋”模型,它类似于化学键的海特勒-伦敦(Heitler-London)模型。磁性被认为是完全来自均匀分布在晶体中的电子自旋。这个模型可以概括的范围包括磁性电子的巡回效应,即导带结构,或者考虑到在大多数铁磁材料中实际上每个格点的自旋数不是整数的事实,这在第\ref{sec:11}章中进行了简要介绍。在我们看来,对此类情况的扩展要么涉及很广,要么仅具有定性有效性。因此在某种意义上,传统的自旋波理论只是学术性的,因为它基于过于理想化的模型,但是它同样适用于非导体的反铁磁或铁氧体。

如果不考虑模型的适用性,那么我们一开始就应该对能够从自旋波理论获得的结果进行评估。在铁磁学中,可以得出以下结果:接近绝对零度时的磁化强度(也许是最著名的结果),磁化强度对场强的依赖性,低温下的交换比热以及铁磁共振的频率。 在通常的自旋波处理中被忽略,但是仍基于理想 Heisenberg 或 Heitler-London 模型的项的引入,为自旋-自旋弛豫时间的存在提供了一种机制。自旋波理论也给反铁磁提供了相应结果,尽管这里的近似在严谨性方面值得商榷。

有关自旋波理论的文献很多。本文中大概包含五十篇参考文献,即使如此我们的文献列表可能也是不完整的。已经有两种不同的方法来介绍自旋波的概念,即量子力学方法和半经典方法。量子力学方法是斯莱特(Slater)和布洛赫(Bloch)在试图推导铁磁晶体低能级的近似表达式时提出的。后来由荷斯坦(Holstein)和普里马科夫(Primakoff)引入另一种方法,采用了不同方式推导出基本相同的结果。半经典方法是由海勒(Heller)和克雷默斯(Kramers)提出的,目的是对斯莱特(Slater)和布洛赫(Bloch)引入的自旋波给出一种经典解释。这两种方法最近都被凯弗(Keffer)、卡普兰(Kaplan)和亚菲特(Yafet)在铁磁和反铁磁共振的研究中检验了。

我们现在用数学术语来描述我们使用的局域自旋模型。如果我们像布洛赫原理那样,仅考虑各向同性自旋耦合,那么自旋系统的哈密顿量是
\begin{equation} \label{eq:1}
\mathcal{H}=-Hg\beta\sum\nolimits_iS_{iz}-2J\sum\nolimits_{\mathit{nei}}\mathbf{S}_{i}\cdot \mathbf{S}_{j},
\end{equation}
其中$\beta$是玻尔磁子$e\hslash/2mc$,$g$是朗德因子,约等于$2$,而$\mathbf{S}_i$是原子$i$的自旋角动量,为$\hslash$的倍数,并假设所有的磁性原子有相同的自旋量子数$S$。公式\eqref{eq:1}中的第一项是自旋在一个大小为$H$的磁场中的塞曼能,我们假设磁场方向沿着$z$轴。第二项为交换能;关于交换能与两个自旋矢量的标量积成正比的证明是十分标准的,我们在此就忽略了。我们约束研究对象为简单晶格,并假设除了最近邻原子之间,其他交换能可以忽略。现在所有没有消失的交换积分都等于$J$,并且式\eqref{eq:1}中仅对最邻近原子求和,正如全文习惯使用的下标$\mathit{nei}$所示。耦合常数$J$等于最邻近原子的交换积分,对于铁磁体为正数,对于反铁磁为负数。在更普遍的情况中,交换能的形式为$-2\sum_{j>i}J_{ij}\mathbf{S}_i\cdot\mathbf{S}_j$,并且无论相对距离的大小,对晶体中所有的原子对求和。如果晶格是规则的并且不考虑边界效应,这仍然可以当作自旋波来处理,但是对于这种情况,我们将不给出明确的推导;这个过程其实类似于我们现在所用的方法。

自旋波的所有计算均基于矢量或海森堡模型,其中耦合常数与标量积$\mathbf{S}_{i}\cdot\mathbf{S}_{j}$成正比;伊辛模型则任意地使用了一个势能$S_{iz}S_{jz}$。从数学的角度来看,这是一种简化,因为每个单独的$S_{iz}$是运动的常数,而且可以被量化。但是,量子力学的交换效应实际上导致了矢量形式。因此,伊辛模型是纯粹学术化抽象。实际上,它根本不给出任何自旋波,而自旋波理论最重要的是处理比伊辛模型更真实的情况。

除了式\eqref{eq:1}中包含的各向同性耦合(这么称呼是因为标量积$\mathbf{S}_{i}\cdot\mathbf{S}_{j}$仅依赖于矢量$\mathbf{S}_i$和$\mathbf{S}_j$之间的角度,与他们合成之后的方向无关),实际上存在各向异性相互作用,其中最低次项是“偶极”或者“张量”结构
\begin{equation} \label{eq:2}
\sum\nolimits_{j>i}D_{\mathit{ij}}[(\mathbf{S}_i\cdot\mathbf{S}_j)-3(\mathbf{S}_i\cdot\mathbf{r}_{\mathit{ij}})(\mathbf{S}_j\cdot\mathbf{r}_{\mathit{ij}})/r_{\mathit{ij}}^2].
\end{equation}
对于经典电磁学,即真实的磁性自旋-自旋相互作用,常数$D_{ij}$的值为
\begin{equation} \label{eq:3} 
D_{\mathit{ij}}=g^2\beta^2/r_\mathit{ij}^3.
\end{equation}
然而,现在普遍认为$D_{ij}$的非经典值以及在很短原子间距离处有更大的数量级,可能是由于自旋轨道耦合的间接影响,而这是由复杂的扰动机制引起的,我们在此不作描述。这样的$D_{ij}$非经典值经常被认作是“假偶极”或者“各向异性交换”耦合。最重要的经典偶极耦合效应是长程的。因此当\eqref{eq:3}式中的$D_{ij}$带入\eqref{eq:2}式时,重要的是求和将不受限制,不再局限于最近邻原子。另一方面,$D_{ij}$的非经典部分相对而言是短程的,就像$J_{ij}$。

公式\eqref{eq:2}中表达了两个原子(如果它们的自旋是$\frac{1}{2}$,并且关于它们中心连线对称)间最普遍的各向异性耦合。对于自旋大于$\frac{1}{2}$,还有四极耦合项,如
\begin{equation*}
E_\mathit{ij}(\mathbf{S}_i\cdot\mathbf{r}_\mathit{ij})^2(\mathbf{S}_j\cdot\mathbf{r}_\mathit{ij})^2.
\end{equation*}
对于更大的自旋值更高阶项还会出现。四极相互作用对于各向异性来说是重要的,因为它能在一阶近似下给出简单立方晶体中的各向异性。然而,\eqref{eq:2}式中的偶极结构相互作用,尽管它甚至能够在一阶近似下给出非简单立方晶体中的各向异性,但是只有在二阶近似下才能在简单立方材料给出相应的结果。这种属性是对称性要求的直接结果,因为当所有三个轴都相等时,方向余弦的二次形式(例如基于式\eqref{eq:2}的任何一阶计算得出的结果)会退化为球对称关系。另一方面,当平方时,如在二阶微扰计算中那样,可以获得立方各向异性的四次角依赖$\lambda^2\mu^2+\lambda^2\nu^2+\mu^2\nu^2$特性。

尽管我们可能在任何计算中均不包含四极耦合,但它可能会隐含在本文后面的某些部分中使用的各向异性场中。为了简单起见,我们推迟到第\ref{sec:5}章再包含偶极相互作用\eqref{eq:2},这对于铁磁共振和退磁效应很重要。



%%===========================================================
%%===========================================================

\section{自旋-偏差量子数} \label{sec:2}

继荷斯坦(Holstein)和普里马科夫(Primakof)之后,我们引入了自旋-偏差量子数$n=S-m$,它代替了与原子自旋$S$的$z$分量相对应的普通量子数$m$,代表$S_z$对其最大值$S$的偏离。$n$的允许值为$0,~1,~2,~\cdots,~2S$,它们是自旋-偏差算符$n=S-S_z$的本征值。在该算符$n$为对角矩阵的表示中,$S$的三个分量的不为零的矩阵元由熟悉的表达式给出
\begin{eqnarray} \label{eq:4}
&&\bra{n}S_x\ket{n+1}=\bra{n+1}S_x\ket{n}=\tfrac{1}{2}(n+1)^{\frac{1}{2}}(2S-n)^{\frac{1}{2}};\nonumber\\
&&\bra{n}S_y\ket{n+1}=\bra{n+1}S_y\ket{n}^*=-\tfrac{1}{2}i(n+1)^{\frac{1}{2}}(2S-n)^{\frac{1}{2}};\\
&&\phantom{***}\bra{n}S_z\ket{n}=S-n.\nonumber
\end{eqnarray}
$\mathbf{S}_i$分量的矩阵相对于$n_i$具有\eqref{eq:4}式的形式,但相对于晶体中所有其他原子的$n_j$是对角的。因此,矩阵$\mathbf{S}_i$是公式\eqref{eq:4}和$N-1$个单位矩阵的乘积。

在绝对零度时的平衡态,铁磁晶体的所有自旋平行于外场,所以没有自旋偏差并且所有的$n_i$都消失了。众所周知,如果在这种状态下引入一个自旋偏差到一个特定的原子上,则该自旋偏差不会局限在那个原子上:由于与周围原子的交换相互作用,它将自发地通过晶格传播,从而构成一个“自旋波”,或者更确切地说是一个自旋波包。一个自旋波对应于一个具有确定波矢的自旋偏差的传播。正如我们后面展示的,只要在整个晶格中只有一个自旋偏差,与这个确定波矢的自旋波激发相对应的自旋系统的状态正是哈密顿量\eqref{eq:1}的本征态。总之,如果我们使用撇上标去表记整个晶体的自旋,即矢量求和每个原子的自旋$\mathbf{S}'=\sum_i\mathbf{S}_i$,则\eqref{eq:1}式的本征值可以就被计算:$S_z'=NS$和$S_z'=NS-1$,其中$N$是晶体中的原子总数。

整个自旋波近似的本质在于,对于较小的总晶体自旋偏差,可以将$S_z'=NS-n$状态与完全平行的$S_z'=NS$状态之间的能量差视为大约等于$n$个严格计算得出的单位偏差之和。
换句话说,可加性的假设是与自旋反转有关的。实际上,这种假设是不准确的。例如,如果存在两个自旋偏差,则自旋系统的能量将取决于自旋偏差是否位于(i)同一原子上(仅对于$S>\frac{1}{2}$才可能存在),(ii)相邻原子上,或者(iii)相距更远的原子上(它们的自旋在哈密顿量\eqref{eq:1}中没有耦合)。由于总能量对两个自旋偏差之间距离的依赖性,以及自旋波之间的相互作用,因此不能精确地满足叠加原理。存在两种不同的错误原因。如果我们尝试可视化由此产生的能量修正,则一种修正可以被解释为排斥作用,另一种可以被解释为自旋偏差之间的吸引作用。例如,对于$S=\frac{1}{2}$,排斥是由于给定的某个原子上的自旋偏差不能超过一个。因此,如果两个自旋偏差接近相同的某个晶格位置,由于这种“相互作用”,它们将被散射开。如果$S>\frac{1}{2}$,则仅当存在大于$2S$的自旋波时才会出现这种排斥,然后情况会变得更加复杂。由于两个自旋偏差位于最近邻原子上时的总交换能比位于两个距离更远原子上时的低,因此产生了相互之间的吸引力。在一维晶格中,这种吸引力会导致“自旋复合体”(参见Bethe),即自旋偏差相互绑定的状态,当然也与散射相关联。但是,应该清楚地理解,总的散射截面和由此产生的能量修正由排斥作用和吸引相互作用共同决定。

自旋波近似的现实基础是,在足够低的温度下,自旋偏差之间相互作用的影响可以被忽略。在低温下,磁化强度稍有偏离饱和磁化强度。与晶体中原子的总数相比,晶体中自旋偏差数的平均值较小,因此它们之间的相互作用可以忽略不计。



%%===========================================================
%%===========================================================

\section{铁磁体的简谐近似} \label{sec:3}

自旋矩阵\eqref{eq:4}与众所周知的简谐振子坐标和动量矩阵之间的相似性是当前处理自旋波方法的起点。本章解释了这种方法如何被用于没有偶极修正\eqref{eq:2}的铁磁体情况。反铁磁体的情况将在第\ref{sec:7}-\ref{eq:8}章中进行讨论。

考虑一个质量为$m$角频率为$\omega$的线性谐振子。这个谐振子的坐标$x$和动量$p$中不为零的矩阵元可以写作
\begin{equation} \label{eq:5}
\begin{array}{l}
\bra{n}x\ket{n+1}=\bra{n+1}x\ket{n}=(\hslash/2m\omega)^{\frac{1}{2}}(n+1)^{\frac{1}{2}},\\
\bra{n}p\ket{n+1}=\bra{n+1}p\ket{n}^*=-i(\hslash m\omega/2)^{\frac{1}{2}}(n+1)^{\frac{1}{2}},
\end{array}
\end{equation}
其中$n$现在对应于谐振子激发的量子数,即,
\begin{equation} \label{eq:6}
\left<n\left|\frac{1}{2m}p^2+\frac{1}{2}m\omega^2x^2-\frac{1}{2}\hslash\omega\right|n\right>=n\cdot\hslash\omega.
\end{equation}
当我们引入两个无量纲的变量
\begin{equation} \label{eq:7}
Q=(m\omega/\hslash)^{\frac{1}{2}}x\text{~~和~~}P=(\hslash m\omega)^{-\frac{1}{2}}p,
\end{equation}
我们可以把\eqref{eq:5}式和\eqref{eq:6}式改写成下面的形式:
\begin{eqnarray} \label{eq:8}
&&\bra{n}S^{\frac{1}{2}}Q\ket{n+1}=\bra{n+1}S^{\frac{1}{2}}Q\ket{n}=\tfrac{1}{2}(n+1)^{\frac{1}{2}}(2S)^{\frac{1}{2}};\nonumber\\
&&\bra{n}S^{\frac{1}{2}}P\ket{n+1}=\bra{n+1}S^{\frac{1}{2}}P\ket{n}^*=-\tfrac{1}{2}i(n+1)^{\frac{1}{2}}(2S)^{\frac{1}{2}};\\
&&\bra{n}S-\tfrac{1}{2}(P^2+Q^2-1)\ket{n}=S-n.\nonumber
\end{eqnarray}

将这些表达式与\eqref{eq:4}式中自旋分量的矩阵元相比较,我们看到两组矩阵之间有惊人的相似性。两组矩阵仅在对角线上或沿对角线的方向具有不消失的矩阵元,并且这些矩阵元的值密切相关。然而,还是有两个重要的区别,即:

(i)矩阵\eqref{eq:8}是无限维度的,而\eqref{eq:4}式中的自旋矩阵只有$(2S+1)$-维。或者换一种说法,自旋矩阵\eqref{eq:4}和矩阵\eqref{eq:8}一样都是无限维度的,但是其中所有$n>2S$的矩阵元都等于$0$而不是等于式\eqref{eq:8}中的值。

(ii)自旋矩阵\eqref{eq:4}中出现的因子$(2S-n)^\frac{1}{2}$在简谐振子矩阵\eqref{eq:8}中被$(2S)^\frac{1}{2}$替换。我们注意到,对于任何$S$值,式\eqref{eq:4}中连接前两个态$n=0$和$n=1$的矩阵元和\eqref{eq:8}式中的相等。尽管相比于$2S$,$n$很小,但当$n$取较大值时对应的矩阵元大约都相等。

现在我们用简谐振子矩阵\eqref{eq:8}来代替\eqref{eq:1}式的自旋系统哈密顿量中的自旋矩阵\eqref{eq:4},也就是说,让我们来做以下替换:
\begin{equation} \label{eq:9}
S_x=S^{\frac{1}{2}}Q;~~S_y=S^{\frac{1}{2}}P;~~S_z=S-\tfrac{1}{2}(P^2+Q^2-1).
\end{equation}
我们就得到了以下哈密顿量表达式:
\begin{eqnarray} \label{eq:10}
&\mathcal{H}=-Hg\beta\sum\nolimits_i[S-\tfrac{1}{2}(P_i^2+Q_i^2-1)]\nonumber\\
&\phantom{~~~~}-2JS\sum\nolimits_{\mathit{nei}}(P_iP_j+Q_iQ_j)+\nonumber\\
&\phantom{~~~~~~~}-2J\sum\nolimits_{nei}[S-\tfrac{1}{2}(P_i^2+Q_i^2-1)]\nonumber\\
&\phantom{~~~~~~~~~~~~~~~~~~~~~~~~~~~~~~~~~~~~~~~~~~~~~~~~~~~~}\times[S-\tfrac{1}{2}(P_j^2+Q_j^2-1)].
\end{eqnarray}

尽管温度很低以至于每个谐振子的$n_i\geqslant 2$态并没有被适当的激发,\eqref{eq:10}式中的哈密顿量描述的耦合简谐振子系统的性质和\eqref{eq:1}式中的自旋系统的性质是相同的。然而在更高温度时,偏差就出现了。对于$S=\frac{1}{2}$,当适当地激发$n\geqslant 2$的能级时,忽略排斥相互作用是不合理的,而对于较大的$S$值,会产生附加修正,因为在较高的温度下,$(2S-n)^\frac{1}{2}$因子可能不再由$(2S)^\frac{1}{2}$代替。谐振子系统\eqref{eq:8}的性质与自旋系统\eqref{eq:1}性质之间的偏差是自旋变量“非谐性”的结果,即自旋矩阵\eqref{eq:4}与简谐振子矩阵\eqref{eq:8}之间的偏差导致的。

可以通过在$P$和$Q$中添加高阶(非协性)项来改进\eqref{eq:9}式的近似,从而不仅使对应于$n=0$和$n=1$的矩阵元一致,而且使对应于$n=2$的矩阵元一致等等,\eqref{eq:9}式中的每个附加项都考虑到了与下一个较高$n$值相对应的下一组矩阵元。我们将说明$S=\frac{1}{2}$时的过程。当我们引入量$S^\pm=S_x\pm iS_y,~a=2^{-\frac{1}{2}}(Q+iP)$和$a^*=2^{-\frac{1}{2}}(Q-iP)$时,我们可以写出\eqref{eq:9}式一阶近似的形式
\begin{equation} \label{eq:11}
S^+=(2S)^{\frac{1}{2}}a;~~S^-=(2S)^{\frac{1}{2}}a^*;~~S_z=S-a^*a.
\end{equation}
\eqref{eq:11}式中的右手边的前四个矩阵元(在态$n=0$和$n=1$之间的那些元素)等于左手边的对应矩阵元。在下一级近似中,我们可以替换\eqref{eq:11}式来使对应于态$n=0,~1,$和$2$的九个矩阵元相一致
\begin{equation} \label{eq:12}
\begin{array}{c}
S^+=(2S)^{\frac{1}{2}}(a-a^*aa);~~S^-=(2S)^{\frac{1}{2}}(a^*-a^*a^*a);\\
S_z=S-a^*a+a^*a^*aa;
\end{array}
\end{equation}
我们可以通过写出\eqref{eq:12}中的矩阵来很简单的验证。对于$S>\frac{1}{2}$,可以以类似的方式考虑出现在$n\leqslant 2S$处的因子$(1-n/2S)^\frac{1}{2}$和现在出现在$n>2S$处的斥力。在文献中,已经报道了许多尝试通过展开平方根$(1-n/2S)^\frac{1}{2}$为$n/2S$的级数并且仅保留前几项来改进\eqref{eq:11}式中的近似,但是忽略了$n>2S$时自旋矩阵\eqref{eq:4}的消失所对应的“排斥”相互作用。

这样,对应于$n\leqslant 2S$的所有矩阵元都被部分修正(这仅对$S>\frac{1}{2}$有意义),而在本方法中(也适用于$S=\frac{1}{2}$)逐次近似\eqref{eq:11},\eqref{eq:12}等等,使连续的矩阵元具有正确的值\eqref{eq:4}。但是应注意,不确定该近似过程是否收敛。由于我们目前仅对常规自旋波近似感兴趣,因此在此不再讨论这一点。

通常的自旋波近似不仅忽略自旋变量的非谐性(即使用一阶近似\eqref{eq:9})而且还忽略自旋偏差间的附加吸引作用。这种相互作用是由哈密顿量\eqref{eq:10}中最后一项中的四次项表示的。在$J>0$时公式\eqref{eq:10}中的四次方项前是负号的事实,在物理上可以被解释为一种自旋偏差间的相互吸引作用。当这些项被忽略时,\eqref{eq:10}式中的哈密顿量变成
\begin{eqnarray} \label{eq:13}
&&\mathcal{H}=E_0'+g\beta(H+H_E)\sum\nolimits_i\tfrac{1}{2}(P_i^2+Q_i^2-1)+\nonumber\\
&&\phantom{~~~~~~~~~~~~~~~~~~~~~~~~~~~~~~~~~~~~}-2JS\sum\nolimits_{\mathit{nei}}(P_iP_j+Q_iQ_j),
\end{eqnarray}
其中
\begin{equation*}
H_E=2JSz/g\beta.
\end{equation*}
这里,并且贯穿全文,$z$代表给定原子的最近邻原子数。$H_E$是我们熟悉的饱和分子场,上式中的常数为
\begin{equation} \label{eq:14}
E_0'=-NHg\beta S-\tfrac{1}{2}Nz\cdot 2JS^2,
\end{equation}
是所有自旋平行于外场的完全饱和状态下的能量。

从\eqref{eq:13}式中我们可以看到,公式\eqref{eq:1}中的哈密顿量现在简化为$N$个由(简化的)坐标和动量 $Q_i$和$P_i$描述的耦合简谐振子系统的哈密顿量。除了\eqref{eq:13}式的最后一项表示相邻协振子之间的耦合(仔细地将其与相邻自旋偏差之间的相互作用加以区别),表达式\eqref{eq:13}的形式正如我们现在将要展示的那样,该表达正是我们期望的形式。因为我们忽略了自旋偏差间的相互作用,我们可能假设每个自旋偏差都被没有自旋偏差的原子包围。那么在一个特定原子上有一个自旋偏差的状态与在那个原子上没有自旋偏差的相应状态之间的能量差就由下式给出:
\begin{eqnarray} \label{eq:15}
&&\Delta E=[-Hg\beta(S-1)-2JzS(S-1)]\nonumber\\
&&\phantom{~~~~~~~~~~~~~~~~~~~}-[-Hg\beta S-2JzS^2]=g\beta(H+H_E),
\end{eqnarray}
如果我们忽略了对应于\eqref{eq:13}式中的耦合项的自旋$x$和$y$分量的影响。当我们想要用一个简谐振子代替自旋,我们必须要求简谐振子的第一激发能等于引入一个孤立的自旋偏差所需要的能量,即必须使谐振子的$\hslash\omega$等于\eqref{eq:15}式中的能量。再次忽略临近简谐振子之间的耦合,我们可以得到哈密顿量
\begin{equation} \label{eq:16}
\mathcal{H}_0=\sum_i(\frac{1}{2m}p_i^2+\frac{1}{2}m\omega^2x_i^2-\frac{1}{2}\hslash\omega)+E_0',
\end{equation}
其中$E'_0$由\eqref{eq:14}式给出并且等于所有谐振子处于基态时的能量。当我们根据\eqref{eq:7}式调整坐标和动量,并使$\hslash\omega$等于\eqref{eq:15}式中的能量时,表达式\eqref{eq:16}与\eqref{eq:13}相同,除了耦合项,即涉及对临近原子的求和项,消失了。在伊辛模型中,该项毫无疑问地被忽略,但是对于本文或自旋波的任何其他处理而言,这都是非常重要的。

寻找耦合谐振子系统的简正模式的问题是一个简单的问题,所需的变换是众所周知的斯拉特(Slater)变换,无论是对于行波
\begin{equation} \label{eq:17}
\begin{array}{l}
Q_k=(1/N)^{\frac{1}{2}}\sum\nolimits_{i}\exp(i\mathbf{k}\cdot\mathbf{r}_i)Q_i;\\
P_k=(1/N)^{\frac{1}{2}}\sum\nolimits_{i}\exp(-i\mathbf{k}\cdot\mathbf{r}_i)P_i;
\end{array}
\end{equation}
或者真正的驻波
\begin{subequations} \label{eq:18}
\begin{equation}
k_x \geqslant 0 \left\{
\begin{array}{l}
Q_k=(2/N)^{\frac{1}{2}}\sum\nolimits_{i}\cos(\mathbf{k}\cdot\mathbf{r}_i)Q_i;\\
P_k=(2/N)^{\frac{1}{2}}\sum\nolimits_{i}\sin(\mathbf{k}\cdot\mathbf{r}_i)P_i;
\end{array}
\right.
\end{equation}
\begin{equation}
k_x < 0 \left\{
\begin{array}{l}
Q_k=(2/N)^{\frac{1}{2}}\sum\nolimits_{i}\sin(\mathbf{k}\cdot\mathbf{r}_i)Q_i;\\
P_k=(2/N)^{\frac{1}{2}}\sum\nolimits_{i}\cos(\mathbf{k}\cdot\mathbf{r}_i)P_i.
\end{array}
\right.
\end{equation}
\end{subequations}
$\mathbf{k}$值的选择需要一些说明。通常会引入所谓的周期性或冯·卡门边界条件,这意味着波函数在穿过晶格后会恢复为其原始值。恰如其分地表达是,这种边界条件的施加意味着在晶体最边缘的自旋仍有临近原子,而不因为处在一侧边缘就没有临近原子。做出这种不真实的假设是为了将所有自旋情况放在同一个水平上,从而使该问题具有充分的周期性,从而简化了分析过程。只要晶体很大,使用这种边界条件所涉及的误差就可以忽略不计,除非存在长程作用力。(然而偶极相互作用可能会导致麻烦,因为它仅以$1/r^3$衰减)。在周期性边界条件下,矢量$\mathbf{k}/2\pi$的允许值通常由晶体所谓的倒格子中的格点表示。对于简单立方晶格和边长为$L$的立方晶体,$k_x,~k_y,$和$k_z$的允许值是$2\pi/L$的整数倍。

就驻波幅度\eqref{eq:18}而言,公式\eqref{eq:13}中的哈密顿量有如下形式
\begin{equation} \label{eq:19}
\mathcal{H}=\sum\nolimits_k\tfrac{1}{2}(P_k^2+Q_k^2-1)\hslash\omega_k+E_0',
\end{equation}
其中
\begin{equation} \label{eq:20}
\hslash\omega_k=Hg\beta+2JS[z-\sum\nolimits_a\cos(\mathbf{k}\cdot\mathbf{a})],
\end{equation} 
求和遍及$z$个矢量$\mathbf{a}$,矢量$\mathbf{a}$连接某个原子核其$z$个临近原子。

根据式\eqref{eq:18}和\eqref{eq:7}中的定义,很容易就可以证明算符的本征值为
\begin{equation*}
\mathfrak{n}_k=\tfrac{1}{2}(P_k^2+Q_k^2-1),
\end{equation*}
是$0,~1,~2,~\cdots$,所以式\eqref{eq:19}中的哈密顿量的本征值由下式给出
\begin{equation} \label{eq:21}
E(\mathfrak{n}_k)=\sum\nolimits_k\mathfrak{n}_k\cdot\hslash\omega_k+E_0',
\end{equation}
其中,简正模式$\mathbf{k}$的频率由式\eqref{eq:20}的色散关系给出。我们使用德语字母$\mathfrak{n}_k$来指定与晶体的各种自旋波相关的量子数,以便将它们与自旋偏差算符或单个原子的量子数区分,我们用$n_i$表示,当包括原子间的交换耦合时,它们不是好的量子数。$\mathfrak{n}_k$或$n_i$每增加一个单位,晶体在$z$方向上的总自旋便减少一个单位的,而$\sum_k\mathfrak{n}_k$或$\sum_i n_i$都可以很好地给出晶体的总旋转偏差。


公式\eqref{eq:20}和\eqref{eq:21}中的结果是众所周知的自旋波近似下自旋系统\eqref{eq:1}低能级的表达式。可用于求出低温下铁磁自旋系统的配分函数。然而,只有在温度很低以至于可以用$k$的级数展开色散关系\eqref{eq:20}的右边部分并且仅保留二次项时,该计算才是可行的。如果我们这样做,式\eqref{eq:20}变成
\begin{equation} \label{eq:22}
\hslash\omega_k=Hg\beta+JS\sum\nolimits_aa^2k^2\cos^2\theta_{k,a},
\end{equation}
其中$\theta_{k,a}$是矢量$\mathbf{a}$和$\mathbf{k}$间的夹角。对于任何立方阵列,$\cos^2\theta_{k,a}$可以用其平均值$\frac{1}{3}$代替。此外,对于一个简单、体心或者面心晶格,我们有$za^2=6l^2$,其中$l$是初基原胞的长度,所以
\begin{equation} \label{eq:23}
\hslash\omega_k=Hg\beta+2JSk^2l^2.
\end{equation}



%%===========================================================
%%===========================================================

\section{自旋波近似下铁磁体的磁化强度和比热} \label{sec:4}

通常通过设置配分函数,并注意自旋波基本上对应于波色-爱因斯坦统计的各种激发可能性,从式\eqref{eq:22}获得磁化强度对温度的依赖性。但是,我们可以通过深入使用简谐振子的形式更简单地获得结果。一个角频率为$\omega_k$的简谐振子的能级,除了基态时的半量子数,对于连续的量子数$\mathfrak{n}_k=1,~2,~\cdots$分别为$\hslash\omega_k,~2\hslash\omega_k,~\cdots$。众所周知的相应平均能量是
\begin{equation} \label{eq:24}
\bar{E}_k=\frac{\hslash\omega_k}{\exp(\hslash\omega_k/kT)-1}.
\end{equation}
没有自旋波,对$\mathfrak{n}_k=1,~2,~\cdots$自旋偏差数分别是$1,~2,~\cdots$。因此谐振子$\omega_k$的平均能量和对应的平均自旋偏差仅仅差了一个因子$\hslash\omega_k$。则不同自旋波的总自旋偏差为
\begin{equation} \label{eq:25}
\braket{NS-S_z'}_{A_V}=\sum\nolimits_k[\exp(\hslash\omega_k/kt)-1]^{-1},
\end{equation}
其中求和遍及$\mathbf{k}$-晶格中所有允许的格点



\begin{eqnarray} \label{eq:26}
&&M_{z0}-M_{z}=g\beta\left(\frac{L}{2\pi}\right)^3\nonumber\\
&&\phantom{~~~~~~~}\times\iiint\frac{\mathrm{d}k_x\mathrm{d}k_y\mathrm{d}k_z}{\exp(2JS(k_x^2+k_y^2+k_z^2)l^2/kT)-1},
\end{eqnarray}

\begin{equation} \label{eq:27}
\begin{array}{l}
M_{z0}-M_z\\
\phantom{~~~}=g\beta\displaystyle\left(\frac{L}{2\pi}\right)^3 4\pi\int_0^\infty\frac{k_0^2\mathrm{k_0}}{\exp(2JSk_0^2l^2/kT)-1}\\
\phantom{~~~}=\displaystyle\left(\frac{L}{l}\right)^3\frac{g\beta}{2\pi^2}\left(\frac{kT}{2JS}\right)^{\frac{3}{2}}\int_0^\infty\frac{V^2}{e^{V^2}-1}\mathrm{d}V\\
\phantom{~~~}=\displaystyle\left(\frac{L}{l}\right)^3\frac{g\beta}{2\pi^2}\left(\frac{kT}{2JS}\right)^{\frac{3}{2}}\int_0^\infty V^2(e^{-V^2}+e^{-2V^2}+\cdots)\mathrm{d}V\\
\phantom{~~~}=\displaystyle\left(\frac{L}{l}\right)^3\frac{g\beta}{2\pi^2}\left(\frac{kT}{2JS}\right)^{\frac{3}{2}}\frac{\pi^{\frac{1}{2}}}{4}\left(1+\frac{1}{2^{\frac{3}{2}}}+\frac{1}{3^{\frac{3}{2}}}+\cdots\right).
\end{array}
\end{equation}

\begin{equation} \label{eq:28}
\begin{array}{l}
M_z=M_{z0}\displaystyle\left[1-\left(\frac{kT}{2JS}\right)^{\frac{3}{2}}\left(\frac{1}{4\pi}\right)^{\frac{3}{2}}\frac{1}{S}\zeta(\frac{3}{2})\right]\\
\phantom{~~~~}=\displaystyle M_{z0}\left[1-0.1187\left(\frac{kT}{2JS}\right)^{\frac{3}{2}}\frac{1}{S}\right].
\end{array}
\end{equation}

\begin{equation} \label{eq:29}
U=\left(\frac{L}{2\pi}\right)^3 4\pi\int_0^\infty\frac{(2JSk_0^2l^2)k_0^2\mathrm{d}k_0}{\exp(2JSk_0^2l^2/kT)-1},
\end{equation}

\begin{eqnarray} \label{eq:30}
&&U=\frac{(L/l)^3(2JS)}{2\pi^2}\left(\frac{kT}{2JS}\right)^{\frac{5}{2}}\nonumber\\
&&\phantom{~~~~~~~~~~~~~~~~~~~~~~~~~~~}\times\int_0^\infty V^4(e^{-V^2}+e^{-2V^2}+\cdots)\mathrm{d}V,
\end{eqnarray}

\begin{equation} \label{eq:31}
C_V=\mathrm{d}U/\mathrm{d}T=cNk(kT/2JS)^{\frac{3}{2}}
\end{equation}

\begin{equation} \label{eq:32}
c=\tfrac{5}{2}\left(\frac{1}{2\pi^2}\right)\frac{3\pi^{\frac{1}{2}}}{8}\zeta(\tfrac{5}{2})=\frac{15}{8\pi^{\frac{3}{2}}}\zeta(\tfrac{5}{2})=0.113.
\end{equation}


%%===========================================================
%%===========================================================

\section{偶极结构相互作用的影响} \label{sec:5}

\begin{eqnarray} \label{eq:33}
&&\mathcal{H}=-2J\sum\nolimits_{\mathit{nei}}\mathbf{S}_i\cdot\mathbf{S}_j+\sum\nolimits_{j>i}D_{ij}[\mathbf{S}_i\cdot\mathbf{S_j}\nonumber\\
&&\phantom{~~~~~~~~~~~~~~~~~~~~~~~~~~~~~~~~~~~~~~~~~~~~~}-3(\boldsymbol{\alpha}_{ij}\cdot\mathbf{S}_i)(\boldsymbol{\alpha}_{ij}\cdot\mathbf{S}_j)].
\end{eqnarray}

\begin{equation} \label{eq:34}
n_i=\tfrac{1}{2}(P_i^2+Q_i^2-1).
\end{equation}


\begin{equation} \label{eq:35}
\mathcal{H}=-2J\sum\nolimits_{\mathit{nei}}\mathbf{S}_i\cdot\mathbf{S}_j+D_0+D_1+D_2+D_3+D_4,
\end{equation}

\begin{eqnarray}
&&D_0=S^2\sum\nolimits_{j>i}D_{ij}(1-3\gamma_{ij}^2),\label{eq:36}\\
&&D_1=-6S^{\frac{3}{2}}\sum\nolimits_{j>i}D_{ij}[\alpha_{ij}\gamma_{ij}Q_j+\beta_{ij}\gamma_{ij}P_j],\label{eq:37}\\
&&D_2=S\sum\nolimits_{j>i}D_{ij}[(1-\alpha_{ij}^2)Q_iQ_j-3\alpha_{ij}\beta_{ij}(Q_iP_j+P_iQ_j)\nonumber\\
&&\phantom{~~~~~~~~~~~~~~~~~~~~~~~~}+(1-3\beta_{ij}^2)P_iP_j-(1-3\gamma_{ij}^2)2n_i],\label{eq:38}\\
&&D_3=3S^{\frac{1}{2}}\sum\nolimits_{j>i}D_{ij}[\alpha_{ij}\gamma_{ij}(Q_i)n_j+n_iQ_j)\nonumber\\
&&\phantom{~~~~~~~~~~~~~~~~~~~~~~~~~~~~~~~~~~~~~~~~~~}+\beta_{ij}\gamma_{ij}(P_in_j+n_iP_j)], \label{eq:39}\\
&&D_4=\sum\nolimits_{j>i}D_{ij}(1-3\gamma_{ij}^2)n_in_j.\label{eq:40}
\end{eqnarray}

\begin{equation} \label{eq:41}
N_z=(V/N)\sum\nolimits_j\boldsymbol{r}_{ij}^{-3}(1-3\gamma_{ij}^2)+(4\pi/3)
,\end{equation}


\begin{equation} \label{eq:42}
M_0=(N/V)g\beta S,
\end{equation}

\begin{equation} \label{eq:43}
D_0=-\tfrac{1}{2}VM_0[(4\pi/3)M_0-N_zM_0].
\end{equation}

\begin{equation} \label{eq:44}
\sum\nolimits_jD_{ij}\alpha_{ij}\gamma_{ij}=\sum\nolimits_jD_{ij}\beta_{ij}\gamma_{ij}=0,
\end{equation}

\begin{equation} \label{eq:45}
-2J\sum\nolimits_{\mathit{nei}}\mathbf{S}_i\cdot\mathbf{S}_j+D_2.
\end{equation}

\begin{eqnarray} \label{eq:46}
&&\tfrac{1}{2}N^{-1}S\sum\nolimits_{i,j}D_{ij}(1-3\alpha_{ij}^2)\sum\nolimits_{k,k'}Q_kQ_{k'}\nonumber\\
&&\phantom{~~~~~~~~~~~~~~~~~~~~~~~~~~~~~~~~~~~~~~~}\times\exp[-i\mathbf{k}\cdot\mathrm{r}_i-i\mathbf{k}'\cdot\mathbf{r}_j],
\end{eqnarray}

\begin{eqnarray} \label{eq:47}
&&\tfrac{1}{2}N^{-1}S\sum\nolimits_{k,k'}Q_kQ_{k'}\sum\nolimits_{i}\exp[-i(\mathbf{k}+\mathbf{k}')\cdot\mathbf{r}_i]\nonumber\\
&&\phantom{~~~~~~~~~~~~~~~~}\times\sum\nolimits_jD_{ij}(1-3\alpha_{ij}^2)\exp[-i\mathbf{k}'\cdot(\mathbf{r}_j-\mathbf{r}_i)].
\end{eqnarray}


\begin{equation} \label{eq:48}
\begin{array}{l}
A_{xx}(\mathbf{k})=S\sum\nolimits_jD_{ij}(1-3\alpha_{ij}^2)\exp[i\mathbf{k}\cdot(\mathbf{r}_j-\mathbf{r}_i)];\\
A_{xy}(\mathbf{k})=-3S\sum\nolimits_jD_{ij}\alpha_{ij}\beta_{ij}\exp[i\mathbf{k}\cdot(\mathbf{r}_j-\mathbf{r}_i)];
\end{array}
\end{equation}

\begin{equation} \label{eq:49}
N^{-1}\sum\nolimits_i\exp(i\mathbf{k}\cdot\mathbf{r}_i)=\delta(\mathbf{k}),
\end{equation}

\begin{equation} \label{eq:50}
\tfrac{1}{2}\sum\nolimits_kA_{xx}(\mathbf{k})Q_kQ_{-k}.
\end{equation}


\begin{eqnarray} \label{eq:51}
&&D_2=\tfrac{1}{2}\sum\nolimits_k[A_{xx}(\mathbf{k})-A_{zz}(0)]Q_kQ_{-k}\nonumber\\
&&\phantom{~~~~~~~~~~~~~~~~~~~}+\tfrac{1}{2}\sum\nolimits_k[A_{yy}(\mathbf{k})-A_{zz}(0)]P_kP_{-k}\nonumber\\
&&\phantom{~~~~~~~~~~~~~~~~~~~~~~~~~~~~~~~}+\sum\nolimits_kA_{xy}(\mathbf{k})Q_kP_k+\tfrac{1}{2}NA_{zz}(0).
\end{eqnarray}

\begin{equation} \label{eq:52}
\mathcal{H}=\tfrac{1}{2}\sum\nolimits_k[A(\mathbf{k})Q_k^2+B(\mathbf{k})P_k^2+2C(\mathbf{k})Q_kP_k]+E_0''.
\end{equation}

\begin{eqnarray} 
&&A(\mathbf{k})=\hslash\omega_k^{(0)}+A_{xx}(\mathbf{k})-A_{zz}(0); \label{eq:53}\\
&&B(\mathbf{k})=\hslash\omega_k^{(0)}+A_{yy}(\mathbf{k})-A_{zz}(0); \label{eq:54}\\
&&C(\mathbf{k})=A_{xy}(\mathbf{k}); \label{eq:55}
\end{eqnarray}

\begin{equation} \label{eq:56}
E_0''=E_0'+\tfrac{1}{2}N(S+1)A_{zz}(0)-\sum\nolimits_k\tfrac{1}{2}\hslash\omega_k^{(0)},
\end{equation}

\begin{equation} \label{eq:57}
\mathcal{H}=\sum\nolimits_k\tfrac{1}{2}(Q_k'^2+P_k'^2)(AB-C^2)^{\frac{1}{2}}+E_0''.
\end{equation}

\begin{equation} \label{eq:58}
E(\mathfrak{n}_k)=\sum\nolimits_k\mathfrak{n}_k\cdot\hslash\omega_k+E_0,~~(\mathfrak{n}=0,1,2,\cdots),
\end{equation}

\begin{equation} \label{eq:59}
\hslash\omega_k=[A(\mathbf{k})B(\mathbf{k})-C(\mathbf{k})^2]^{\frac{1}{2}},
\end{equation}


\begin{equation} \label{eq:60}
E_0=E_0''+\tfrac{1}{2}\sum\nolimits_k\hslash\omega_k,
\end{equation}


%%===========================================================
%%===========================================================

\section{铁磁共振吸收的自旋波理论} \label{sec:6}

\begin{equation} \label{eq:61}
\hslash\omega=g\beta\{[H+(N_x-N_z)M][H+(N_y-N_z)M]\}^{\frac{1}{2}}.
\end{equation}

\begin{equation} \label{eq:62}
\mathcal{H}'=-H_zg\beta\sum\nolimits_iS_{ix},
\end{equation}

\begin{equation} \label{eq:63}
\mathcal{H}'=-H_xg\beta S^{\frac{1}{2}}\sum\nolimits_iQ_i=-H_xg\beta(NS)^{\frac{1}{2}}Q_0,
\end{equation}

\begin{equation} \label{eq:64}
\mathcal{H}'=-H_xg\beta(NS)^{\frac{1}{2}}(aQ_0'+bP_0'),
\end{equation}

\begin{equation} \label{eq:65}
\mathfrak{n}_0'=\mathfrak{n}_0\pm 1~~\text{和}~~\mathfrak{n}_k'=\mathfrak{n}_k~~\text{对于}~~k\neq 0.
\end{equation}

\begin{equation} \label{eq:66}
\hslash\omega_0=[A(0)B(0)-C(0)^2]^{\frac{1}{2}},
\end{equation}

\begin{eqnarray} \label{eq:67}
&&\hslash\omega_0=\{[g\beta H+A_{xx}(0)-A_{zz}(0)]\nonumber\\
&&\phantom{~~~~~~~~~~~~~~~}\times[g\beta H+A_{yy}(0)-A_{zz}(0)]-A_{xy}(0)^2\}^{\frac{1}{2}}.
\end{eqnarray}

\begin{eqnarray} \label{eq:68}
&&A_{zz}(0)=S\sum\nolimits_jg^2\beta^2r_{ij}^{-3}(1-3\gamma_{ij}^2)\nonumber\\
&&\phantom{~~~~~~~~~~~~~~~~~~~~~~~~~~~~~~~~}=g\beta N_z(N/V)g\beta S=g\beta N_zM_0,
\end{eqnarray}

\begin{equation} \label{eq:69}
\hslash\omega_0=g\beta\{[H+(N_x-N_z)M_0][H+(N_y-N_z)M_0]\}^{\frac{1}{2}},
\end{equation}

\begin{equation} \label{eq:70}
\mathcal{H}=-VM_zH+\tfrac{1}{2}V(N_xM_x^2+N_yM_y^2+N_zM_z^2).
\end{equation}

\begin{equation} \label{eq:71}
\mathcal{H}=-VM_zH+\tfrac{1}{2}V[(N_x-N_z)M_x^2+(N_y-N_z)M_y^2].
\end{equation}

\begin{eqnarray} \label{eq:72}
&&\mathcal{H}=-Hg\beta J_z+\tfrac{1}{2}(g\beta M/J)\nonumber\\
&&\phantom{~~~~~~~~~~~~~~~~~~~~~~~}\times[(N_x-N_z)J_x^2+(N_y-N_z)J_y^2].
\end{eqnarray}

\begin{equation} \label{eq:73}
J_x=J^{\frac{1}{2}}Q;~~J_y=J^{\frac{1}{2}}P;~~J_z=J-\tfrac{1}{2}(P^2+Q^2-1),
\end{equation}

\begin{equation} \label{eq:74}
\mathcal{H}=\text{const}+g\beta\tfrac{1}{2}(aQ^2+bP^2),
\end{equation}

\begin{equation} \label{eq:75}
Q=(b/a)^{\frac{1}{2}}Q',~~P=(a/b)^{\frac{1}{2}}P',
\end{equation}

\begin{equation} \label{eq:76}
\mathcal{H}=\text{const}+g\beta(ab)^{\frac{1}{2}}\tfrac{1}{2}(P'^2+Q'^2),
\end{equation} 


%%===========================================================
%%===========================================================

\section{反铁磁的简谐近似} \label{sec:7}

\begin{equation} \label{eq:77}
\mathcal{H}=-2J\sum\nolimits_{\textit{nei}}\mathbf{S}_i\cdot\mathbf{S}_j.
\end{equation} 

\begin{equation} \label{eq:78}
\mathcal{H}=2J\sum\nolimits_{\mathit{nei}}\mathbf{S}_i\cdot\mathbf{S}_j,
\end{equation}

\begin{equation} \label{eq:79}
\mathcal{H}=2J\sum\nolimits_{\mathit{nei}}\mathbf{S}_i\cdot\mathbf{S}_j-H_Ag\beta(\sum\nolimits_iS_{iz}-\sum\nolimits_jS_{jz}),
\end{equation}

\begin{eqnarray} \label{eq:80}
&&\mathcal{H}=2J\sum\nolinebreak_{\mathit{nei}}\mathbf{S}_i\cdot\mathbf{S}_j-(H+H_A)g\beta\sum\nolimits_iS_{iz}+\nonumber\\
&&~~~~~~~~~~~~~~~~~~~~~~~~~~~~~~~~~~~~~~~~~~~~~-(H-H_A)g\beta\sum\nolimits_jS_{jz},
\end{eqnarray}

\begin{equation} \label{eq:81}
n_i=S-m_i,~~\text{和}~~n_j=S+m_j.
\end{equation} 

\begin{eqnarray} 
&&S_{ix}=S^{\frac{1}{2}}Q_i;~~S_{iy}=S^{\frac{1}{2}}P_i;~~S_{iz}=S-\tfrac{1}{2}(P_i^2+Q_i^2-1);\label{eq:82}\\
&&S_{jx}=S^{\frac{1}{2}}Q_j;~~S_{iy}=-S^{\frac{1}{2}}P_j;~~S_{jz}=-S+\tfrac{1}{2}(P_j^2+Q_j^2-1).\label{eq:83}
\end{eqnarray} 

\begin{eqnarray} \label{eq:84}
&&\mathcal{H}=E_0'+2JS\sum\nolimits_{\mathit{nei}}(Q_iQ_j-P_iP_j)-2J\sum\nolimits_{\mathit{nei}}n_in_j\nonumber\\
&&~~~~~~~~~~~~~~~~~~~~+[2JS_z+g\beta(H_A+H)]\sum\nolimits_in_i\nonumber\\
&&~~~~~~~~~~~~~~~~~~~~~~~~~~~~+[2JS_z+g\beta(H_A=H)]\sum\nolimits_jn_j,
\end{eqnarray}

\begin{equation} \label{eq:85}
n_i=\tfrac{1}{2}(P_i^2+Q_i^2-1)~~\text{和}~~n_j=\tfrac{1}{2}(P_j^2+Q_j^2-1),
\end{equation}	

\begin{equation} \label{eq:86}
E_0'=-NzJS^2-H_Ag\beta NS.
\end{equation}

\begin{equation} \label{eq:87}
\left\{
\begin{array}{l}
Q_{1k}=(2/N)^{\frac{1}{2}}\sum\nolimits_i\exp(i\mathbf{k}\cdot\mathbf{r}_i)Q_i;\\
P_{1k}=(2/N)^{\frac{1}{2}}\sum\nolimits_i\exp(-i\mathbf{k}\cdot\mathbf{r}_i)P_i;
\end{array}
\right.
\end{equation}

\begin{equation} \label{eq:88}
\left\{
\begin{array}{l}
Q_{2k}=(2/N)^{\frac{1}{2}}\sum\nolimits_j\exp(-i\mathbf{k}\cdot\mathbf{r}_j)Q_j;\\
P_{2k}=(2/N)^{\frac{1}{2}}\sum\nolimits_j\exp(i\mathbf{k}\cdot\mathbf{r}_j)P_j.
\end{array}
\right.
\end{equation}

\begin{eqnarray} \label{eq:89}
&&\mathcal{H}=E_0'+\sum\nolimits_k\{[2JSz+g\beta(H_A+H)]\nonumber\\
&&~~~~~~\times\tfrac{1}{2}(P_{1k}P_{1-k}+Q_{1k}Q_{1-k}-1)\nonumber\\
&&~~~~~~+[2JSz+g\beta(H_A-H)]\tfrac{1}{2}(P_{2k}P_{2-k}+Q_{2k}Q_{2-k}-1)\nonumber\\
&&~~~~~~~~~~~~~~~~~~~~~~~~~~~~~~+2JS\gamma_k(Q_{1k}Q_{2k}-P_{1k}P_{2k})\}.
\end{eqnarray}


\begin{equation} \label{eq:90}
\gamma_k=z^{-1}\sum\nolimits_a\cos(\mathbf{k}\cdot\mathbf{a}),
\end{equation}

\begin{equation} \label{eq:91}
\begin{array}{l}
\left\{
\begin{array}{l}
Q_1k=c_{1k}Q_{1k}'+c_{2k}Q_{2-k}';\\
Q_{2-k}=c_{2k}Q_{1k}'+c_{1k}Q_{2-k}';
\end{array}
\right.\\
\left\{
\begin{array}{l}
P_{1k}=c_{1k}P_{1k}'-c_{2k}P_{2-k}',\\
P_{2-k}=-c_{2k}P_{1k}'+c_{1k}P_{2-k}',
\end{array}
\right.
\end{array}
\end{equation}

\begin{equation} \label{eq:92}
c_{1k}=\rho_k/(\rho_k^2-\gamma_k^2)^\frac{1}{2};~~c_{2k}=-\gamma_k/(\rho_k^2-\gamma_k^2)^\frac{1}{2},
\end{equation}

\begin{equation} \label{eq:93}
\rho_k=1+H_A/H_E+[(1+H_A/H_E)^2-\gamma_k^2]^\frac{1}{2}.
\end{equation}

\begin{equation*}
H_E=2JSz/g\beta,
\end{equation*}

\begin{eqnarray} \label{eq:94}
&&\mathcal{H}=E_0+\sum\nolimits_k\tfrac{1}{2}(P_{1k}'P_{1-k}'+Q_{1k}'Q_{1-k}'-1)\hslash\omega_{1k}\nonumber\\
&&~~~~~~~~~~~~~~~~~~~~+\sum\nolimits_k\tfrac{1}{2}(P_{2k}'P_{2-k}'+Q_{2k}'Q_{2-k}'-1)\hslash\omega_{2k}.
\end{eqnarray}

\begin{equation} \label{eq:95}
E(\mathfrak{n}_{1k},\mathfrak{n}_{2k})=E_0+\sum\nolimits_k(\mathfrak{n}_{1k}\hslash\omega_{1k}+\mathfrak{n}_{2k}\hslash\omega_{2k}),
\end{equation}

\begin{equation} \label{eq:96}
\left\{
\begin{array}{l}
\hslash\omega_{1k}=g\beta[(H_E+H_A)^2-\gamma_k^2H_E^2]^\frac{1}{2}+g\beta H;\\
\hslash\omega_{2k}=g\beta[(H_E+H_A)^2-\gamma_k^2H_E^2]^\frac{1}{2}-g\beta H;
\end{array}
\right.
\end{equation}

\begin{equation} \label{eq:97}
E_0=E_0'+\tfrac{1}{2}\sum\nolimits_k(\hslash\omega_{1k}+\hslash\omega_{2k}),
\end{equation}

\begin{equation} \label{eq:98}
\hslash\omega_{20}=g\beta(H_c-H);~~H_c=[H_A(H_A+2H_E)]^\frac{1}{2}.
\end{equation}

\begin{equation} \label{eq:99}
\left\{
\begin{array}{l}
\hslash\omega_1=g\beta[H_A(H_A+2H_E)]^\frac{1}{2}+g\beta H;\\
\hslash\omega_2=g\beta[H_A(H_A+2H_E)]^\frac{1}{2}-g\beta H;
\end{array}
\right.
\end{equation}


%%===========================================================
%%===========================================================

\section{自旋波近似下的反铁磁基态} \label{sec:8}

\begin{equation} \label{eq:100}
A_{1k}=2^{-\frac{1}{2}}(Q_{1k}'+iP_{1-k}');~~A_{1k}^*=2^{-\frac{1}{2}}(Q_{1-k}'-iP_{1k}'),
\end{equation}

\begin{equation} \label{eq:101}
n_i=(2/N)\sum\nolimits_{k,k'}\exp[i(\mathbf{k}'-\mathbf{k})\cdot\mathbf{r}_i]B_{k'}^*B_k,
\end{equation}

\begin{equation} \label{eq:102}
B_k=c_{1k}A_{1k}+c_{2k}A_{2k}^*.
\end{equation}

\begin{equation} \label{eq:103}
\braket{n_i}=(2/N)\sum\nolimits_i\braket{n_i}=(2/N)\sum\nolimits_k\braket{B_k^*B^*},
\end{equation}

\begin{equation} \label{eq:104}
\braket{n_i}=(2/N)\sum\nolimits_kc_{2k}^2=\Gamma(1+H_A/H_E),
\end{equation}

\begin{equation} \label{eq:105}
\Gamma(x)=(2/N)^\frac{1}{2}\sum\nolimits_k\tfrac{1}{2}[(1-\gamma_k^2/x)^{-\frac{1}{2}}-1],
\end{equation}

\begin{eqnarray} \label{eq:106}
&&\braket{n_i^2}=(2/N)^2\sum\nolimits_{k,k'}\sum\nolimits_{l,l'}\delta(k+l-k'-l')\nonumber\\
&&~~~~~~~~~~~~~~~~~~~~~~~~~~~~~~~~~~~~~~~~~~~\times\braket{B_{k'}^*B_kB_{l'}^*B_l}.
\end{eqnarray}

\begin{equation} \label{eq:107}
\braket{n_i^2}=\Gamma+2\Gamma^2,
\end{equation}

\begin{equation} \label{eq:108}
\braket{n_i^3}=\Gamma+6\Gamma^2+5\Gamma^3.
\end{equation}

\begin{equation} \label{eq:109}
\braket{n_i^m}=p_1+2^mp_2+3^mp_3+\cdots,
\end{equation}

\begin{equation} \label{eq:110}
\left\{
\begin{array}{l}
p_0+p_1+p_2+p_3+\cdots=1;\\
~~~~p_1+2p_2+3p_3+\cdots=\Gamma;\\
~~~~p_1+4p_2+9p_3+\cdots=\Gamma+2\Gamma^2;\\
~~~p_1+8p_2+27p_3+\cdots=\Gamma+6\Gamma^2+5\Gamma^3.\\
~~~~~~~~~~~~~~~\cdots
\end{array}
\right.
\end{equation}

\begin{equation} \label{eq:111}
\left\{
\begin{array}{l}
p_0=1-\Gamma+\Gamma^2-(5/6)\Gamma^3;\\
p_1=\Gamma-2\Gamma^2+(5/2)\Gamma^3;\\
p_2=\Gamma^2-(5/2)\Gamma^3;\\
p_3=(5/6)\Gamma^3.
\end{array}
\right.
\end{equation}

\begin{equation*}
\Gamma(1)=\tfrac{1}{2}(J_D-1),
\end{equation*}

\begin{equation} \label{eq:112}
p_0=0.93~~p_1=0.06;~~p_2=0.01.
\end{equation}


%%===========================================================
%%===========================================================

\section{铁氧体的简谐近似} \label{sec:9}


\begin{eqnarray} \label{eq:113}
&&\mathcal{H}=2J\sum\nolimits_{\mathit{nei}}\mathbf{S}_i\cdot\mathbf{S}_j-(H+H_{A1})g_1\beta\sum\nolimits_iS_{iz}\nonumber\\
&&~~~~~~~~~~~~~~~~~~~~~~~~~~~~~~~~~~~~~~~~~-(H-H_{A2})g_2\beta\sum\nolimits_jS_{jz}
\end{eqnarray}

\begin{equation} \label{eq:114}
n_i=S_1-m_i,~~n_j=S_2+m_j.
\end{equation} 

\begin{eqnarray} \label{eq:115}
&&\mathcal{H}=E_0''+2J(S_1S_2)^\frac{1}{2}\sum\nolimits_{\mathit{nei}}(Q_iQ_j-P_iP_j)-2J\sum\nolimits_\mathit{nei}n_in_j\nonumber\\
&&~~~~~~~~~~~~~~~~~+[2JS_2z+g_1\beta(H+H_{A1})]\sum\nolimits_in_i\nonumber\\
&&~~~~~~~~~~~~~~~~~~~~~~+[2JS_1z+g_2\beta(H-H_{A2})]\sum\nolimits_jn_j,
\end{eqnarray}

\begin{eqnarray} \label{eq:116}
&&E_0''=-NzJS_1S_2-N(g_1S_1-g_2S_2)\beta H+\nonumber\\
&&~~~~~~~~~~~~~~~~~~~~~~~~~~~~-Ng_1S_1\beta H_{A1}-Ng_2S_2\beta H_{A2}.
\end{eqnarray}

\begin{eqnarray} \label{eq:117}
&&\hslash\omega_{1k},\hslash\omega_{2k}\nonumber\\
&&~~~=\{[J(S_1+S_2)z+(K_1+K_2)+\tfrac{1}{2}(g_1-g_2)\beta H]^2\nonumber\\
&&~~~~~~~-4J^2z^2S_1S_2\gamma_k^2\}^\frac{1}{2}\pm[j(S_2-S_1)z\nonumber\\
&&~~~~~~~~~~~~~~~~~~~~~~~~~~~~~~+\tfrac{1}{2}(g_1+g_2)\beta H+(K_1-K_2)],
\end{eqnarray}

\begin{equation} \label{eq:118}
K_1=\tfrac{1}{2}g_1\beta H_{A1};~~K_2=\tfrac{1}{2}g_2\beta H_{A2}.
\end{equation}

\begin{equation} \label{eq:119}
\hslash\omega_{1k}=\frac{g_1S_1-g_2S_2}{S_1-S_2}\beta H+\frac{2JS_1S_2}{S_1-S_2}\sum\nolimits_aa^2k^2\cos\theta_{k,a}.
\end{equation}

\begin{equation} \label{eq:120}
M_z=M_{z0}\left[1-0.1187\left\{\frac{kT(S_1-S_2)}{2JS_1S_2}\right\}^\frac{3}{2}\right],
\end{equation} 

\begin{equation} \label{eq:121}
C_V=cNk\left(\frac{S_1-S_2}{4JS_1S_2}\right)(kT)^\frac{3}{2},
\end{equation}

\begin{equation} \label{eq:122}
\hslash\omega_k=[4J^2S^2z^2(1-\gamma_k^2)]^\frac{1}{2}=2JSka[z\sum\nolimits_a\cos^2\theta_{k,a}]^\frac{1}{2}.
\end{equation}

\begin{equation} \label{eq:123}
\hslash\omega_k=2JSkl(2z)^\frac{1}{2},
\end{equation}

\begin{eqnarray} \label{eq:124}
&&U=2(L/2\pi)^34\pi\int_0^\infty\frac{[2JSk_0l(2z)^\frac{1}{2}]k_0^2\mathrm{d}k}{\exp(2JSk_0l(2z)^\frac{1}{2}/kT)-1}\nonumber\\
&&~~=2(L/l)^3\frac{(kT)^4}{2\pi^2[2JS(2z)^\frac{1}{2}]^3}\\
&&~~~~~~~~~~~~~~~~~~~~~~~~~\times\int_0^\infty V^3(e^{-V}+e^{-2V}+\cdots)\mathrm{d}V.\nonumber
\end{eqnarray}

\begin{equation} \label{eq:125}
C_V=2\left(\frac{L}{l}\right)^3\frac{4!k}{2\pi^2}\left(\frac{kT}{2JS(2z)^\frac{1}{2}}\right)^3\zeta(4).
\end{equation}

\begin{equation} \label{eq:126}
C_V=13.7Nk(kT/12JS)^3,
\end{equation}


%%===========================================================
%%===========================================================

\section{自旋-弛豫过程的简谐振子模型} \label{sec:10}


\begin{equation} \label{eq:127}
\sum\nolimits_{\mathit{ijk}}(a_\mathit{ijk}Q_iQ_jq_k+b_\mathit{ijk}P_iP_jq_k),
\end{equation}


%%===========================================================
%%===========================================================

\section{海特勒-伦敦(Heitler-London)模型以外的自旋波理论的有效性问题} \label{sec:11}


\begin{eqnarray} \label{eq:128}
&&-2\sum\nolimits_{j>i}J_{ij}\mathbf{S}_i\cdot\mathbf{S}_j=-2J\sum\nolimits_{j>i}\mathbf{S}_i\cdot\mathbf{S}_j\nonumber\\
&&~~~~~~~~~~~~~~~~~~~~~~~~=-J[(\sum\nolimits_i\mathbf{S}_i)^2-\sum\nolimits_i(\mathbf{S}_i)^2]\nonumber\\
&&~~~~~~~~~~~~~~~~~~~~~~~~~~~=-J[S'(S'+1)-NS(S+1)].
\end{eqnarray}

\begin{equation*} 
E=-2S\sum\nolimits_\mathit{ij}J_\mathit{ij}\exp(i\mathbf{k}\cdot\mathbf{R}_\mathit{ij})=-2JNS\sum\nolimits_j\exp(i\mathbf{k}\cdot\mathbf{R}_\mathit{ij}).
\end{equation*}

\begin{equation} \label{eq:129}
\left\{
\begin{array}{l}
E=-2JNS~~~\text{对于}~k=0;\\
E=0~~~~~~~~~~~~\text{对于}~k\neq 0.
\end{array}
\right.
\end{equation}

\section*{致谢}
其中一位作者范克兰东克(JVK)希望感谢荷兰纯科研组织(ZWO)的资助,这使得他得以在1953年留在哈佛大学来写作这篇文章的部分内容。

\end{document}







