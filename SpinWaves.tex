\documentclass{article}
\usepackage{braket}
\usepackage{ctex}%add Chinese supports
\usepackage{upgreek}%add special greek symbols
\usepackage{graphicx}%include figure files
\usepackage{wrapfig}
\usepackage{subfigure}
\usepackage{amsmath}%add special math symbols
\usepackage{amssymb}
\usepackage{bm}%bold math
%\usepackage{xfrac}%add special fraction supports
\usepackage{color}
\usepackage[colorlinks, 
linkcolor=black,
anchorcolor=blue, citecolor=blue, urlcolor=blue]{hyperref}% add hypertext capabilities
\usepackage{geometry}
\geometry{a4paper, centering, scale=0.75}

%
%  Created by WW on 11/27/19
%  Copyright © WW. All rights reserved.
%
\usepackage[subfigure]{tocloft}
\renewcommand{\cftsecleader}{\cftdotfill{\cftdotsep}} 

\begin{document}\zihao{5}

\title{{\vspace{-80pt}\normalsize 现代物理评论}~\\\textbf{自旋波}}
\author{范克兰东克(J. Van Kranendonk),范弗莱克(J. H. Van Vleck)~\\哈佛大学,剑桥市,马塞诸塞州}
\date{卷30,第一期 \hfill 一月,1958}
\maketitle

{\vspace{-53pt}\centerline{——————————————————————————————————————————————}}
{\vspace{4pt}\centerline{——————————————————————————————————————————————}}

\vspace{0pt}
\tableofcontents

{\vspace{15pt}\centerline{——————————}\vspace{-10pt}}

%%===========================================================
%%===========================================================

\section{介绍} \label{sec:1}

如何去计算一个铁磁、反铁磁或者铁氧体材料中的磁化强度是一个十分复杂的问题,以至于总是需要采用某种近似。在居里温度以上,配分函数可以展开为交换积分与$kT$之比的幂级数,但是收敛速度非常慢,而且每个后续项的计算难度越来越大。外斯(Weiss)分子场或其衍生方法提供了一种半经验半理论的方式来解决问题,而且如果只作定性考虑,即使在低于居里温度的情况下这种方法也是有效的。外斯-贝特-皮尔斯(Weiss-Bethe-Peierls)方法可以看作是外斯分子场方法的改进,其中群集内的相互作用得到了准确处理,但其与周围环境的耦合仅仅是现象学上的。

极低温下的磁化强度与绝对零度时仅稍有差异,在这种情况下上述方法都不够让人满意。而对于由规则排列的原子组成的晶体,我们有一种自旋波的方法,这正是本论文的主题。我们的目标有两个:首先,将很多分散在不同文献中的结果以一种统一的方式整理在一处;其次,引入了量子力学理论的简化版本。我们的理论基于用一个简谐振子系统来近似磁性自旋系统,与基于产生湮灭算符的传统布洛赫(Bloch)、荷斯坦(Holstein)和普里马科夫(Primakoff)理论的差别很大程度上仅仅是语义学上的。尽管如此,有一些读者,他们觉得简谐振子相比于量子力学场论更加符合直觉,对他们而言这种方法可能更易于理解。

我们的处理方式(除了特别说明的)都是基于磁性固体的海森堡(Heisenberg)或者“局域自旋”模型,它类似于化学键的海特勒-伦敦(Heitler-London)模型。磁性被认为是完全来自均匀分布在晶体中的电子自旋。这个模型可以概括的范围包括磁性电子的巡回效应,即导带结构,或者考虑到在大多数铁磁材料中实际上每个格点的自旋数不是整数的事实,这在第\ref{sec:11}章中进行了简要介绍。在我们看来,对此类情况的扩展要么涉及很广,要么仅具有定性有效性。因此在某种意义上,传统的自旋波理论只是学术性的,因为它基于过于理想化的模型,但是它同样适用于非导体的反铁磁或铁氧体。

如果不考虑模型的适用性,那么我们一开始就应该对能够从自旋波理论获得的结果进行评估。在铁磁学中,可以得出以下结果:接近绝对零度时的磁化强度(也许是最著名的结果),磁化强度对场强的依赖性,低温下的交换比热以及铁磁共振的频率。 在通常的自旋波处理中被忽略,但是仍基于理想 Heisenberg 或 Heitler-London 模型的项的引入,为自旋-自旋弛豫时间的存在提供了一种机制。自旋波理论也给反铁磁提供了相应结果,尽管这里的近似在严谨性方面值得商榷。

有关自旋波理论的文献很多。本文中大概包含五十篇参考文献,即使如此我们的文献列表可能也是不完整的。已经有两种不同的方法来介绍自旋波的概念,即量子力学方法和半经典方法。量子力学方法是斯莱特(Slater)和布洛赫(Bloch)在试图推导铁磁晶体低能级的近似表达式时提出的。后来由荷斯坦(Holstein)和普里马科夫(Primakoff)引入另一种方法,采用了不同方式推导出基本相同的结果。半经典方法是由海勒(Heller)和克雷默斯(Kramers)提出的,目的是对斯莱特(Slater)和布洛赫(Bloch)引入的自旋波给出一种经典解释。这两种方法最近都被凯弗(Keffer)、卡普兰(Kaplan)和亚菲特(Yafet)在铁磁和反铁磁共振的研究中检验了。

我们现在用数学术语来描述我们使用的局域自旋模型。如果我们像布洛赫原理那样,仅考虑各向同性自旋耦合,那么自旋系统的哈密顿量是
\begin{equation} \label{eq:1}
\mathcal{H}=-Hg\beta\sum\nolimits_iS_{iz}-2J\sum\nolimits_{\mathit{nei}}\mathbf{S}_{i}\cdot \mathbf{S}_{j},
\end{equation}
其中$\beta$是玻尔磁子$e\hslash/2mc$,$g$是朗德因子,约等于$2$,而$\mathbf{S}_i$是原子$i$的自旋角动量,为$\hslash$的倍数,并假设所有的磁性原子有相同的自旋量子数$S$。公式\eqref{eq:1}中的第一项是自旋在一个大小为$H$的磁场中的塞曼能,我们假设磁场方向沿着$z$轴。第二项为交换能;关于交换能与两个自旋矢量的标量积成正比的证明是十分标准的,我们在此就忽略了。我们约束研究对象为简单晶格,并假设除了最近邻原子之间,其他交换能可以忽略。现在所有没有消失的交换积分都等于$J$,并且式\eqref{eq:1}中仅对最邻近原子求和,正如全文习惯使用的下标$\mathit{nei}$所示。耦合常数$J$等于最邻近原子的交换积分,对于铁磁体为正数,对于反铁磁为负数。在更普遍的情况中,交换能的形式为$-2\sum_{j>i}J_{ij}\mathbf{S}_i\cdot\mathbf{S}_j$,并且无论相对距离的大小,对晶体中所有的原子对求和。如果晶格是规则的并且不考虑边界效应,这仍然可以当作自旋波来处理,但是对于这种情况,我们将不给出明确的推导;这个过程其实类似于我们现在所用的方法。

自旋波的所有计算均基于矢量或海森堡模型,其中耦合常数与标量积$\mathbf{S}_{i}\cdot\mathbf{S}_{j}$成正比;伊辛模型则任意地使用了一个势能$S_{iz}S_{jz}$。从数学的角度来看,这是一种简化,因为每个单独的$S_{iz}$是运动常数,而且可以被量化。但是,量子力学的交换效应实际上导致了矢量形式。因此,伊辛模型是纯粹学术化抽象。实际上,它根本不给出任何自旋波,而自旋波理论最重要的是处理比伊辛模型更真实的情况。

除了式\eqref{eq:1}中包含的各向同性耦合(这么称呼是因为标量积$\mathbf{S}_{i}\cdot\mathbf{S}_{j}$仅依赖于矢量$\mathbf{S}_i$和$\mathbf{S}_j$之间的角度,与他们合成之后的方向无关),实际上存在各向异性相互作用,其中最低次项是“偶极”或者“张量”结构
\begin{equation} \label{eq:2}
\sum\nolimits_{j>i}D_{\mathit{ij}}[(\mathbf{S}_i\cdot\mathbf{S}_j)-3(\mathbf{S}_i\cdot\mathbf{r}_{\mathit{ij}})(\mathbf{S}_j\cdot\mathbf{r}_{\mathit{ij}})/r_{\mathit{ij}}^2].
\end{equation}
对于经典电磁学,即真实的磁性自旋-自旋相互作用,常数$D_{ij}$的值为
\begin{equation} \label{eq:3} 
D_{\mathit{ij}}=g^2\beta^2/r_\mathit{ij}^3.
\end{equation}
然而,现在普遍认为$D_{ij}$的非经典值以及在很短原子间距离处有更大的数量级,可能是由于自旋轨道耦合的间接影响,而这是由复杂的扰动机制引起的,我们在此不作描述。这样的$D_{ij}$非经典值经常被认作是“赝偶极”或者“各向异性交换”耦合。最重要的经典偶极耦合效应是长程的。因此当\eqref{eq:3}式中的$D_{ij}$代入\eqref{eq:2}式时,重要的是求和将不受限制,不再局限于最近邻原子。另一方面,$D_{ij}$的非经典部分相对而言是短程的,就像$J_{ij}$。

公式\eqref{eq:2}中表达了两个原子(如果它们的自旋是$\frac{1}{2}$,并且关于它们中心连线对称)间最普遍的各向异性耦合。对于自旋大于$\frac{1}{2}$,还有四极耦合项,如
\begin{equation*}
E_\mathit{ij}(\mathbf{S}_i\cdot\mathbf{r}_\mathit{ij})^2(\mathbf{S}_j\cdot\mathbf{r}_\mathit{ij})^2.
\end{equation*}
对于更大的自旋值更高阶项还会出现。四极相互作用对于各向异性来说是重要的,因为它能在一阶近似下给出简单立方晶体中的各向异性。然而,\eqref{eq:2}式中的偶极结构相互作用,尽管它甚至能够在一阶近似下给出非简单立方晶体中的各向异性,但是只有在二阶近似下才能在简单立方材料给出相应的结果。这种属性是对称性要求的直接结果,因为当所有三个轴都相等时,方向余弦的二次形式(例如基于式\eqref{eq:2}的任何一阶计算得出的结果)会退化为球对称关系。另一方面,当平方时,如在二阶微扰计算中那样,可以获得立方各向异性的四次角依赖$\lambda^2\mu^2+\lambda^2\nu^2+\mu^2\nu^2$特性。

尽管我们可能在任何计算中均不包含四极耦合,但它可能会隐含在本文后面的某些部分中使用的各向异性场中。为了简单起见,我们推迟到第\ref{sec:5}章再包含偶极相互作用\eqref{eq:2},这对于铁磁共振和退磁效应很重要。



%%===========================================================
%%===========================================================

\section{自旋-偏差量子数} \label{sec:2}

继荷斯坦(Holstein)和普里马科夫(Primakof)之后,我们引入了自旋-偏差量子数$n=S-m$,它代替了与原子自旋$S$的$z$分量相对应的普通量子数$m$,代表$S_z$对其最大值$S$的偏离。$n$的允许值为$0,~1,~2,~\cdots,~2S$,它们是自旋-偏差算符$n=S-S_z$的本征值。在该算符$n$为对角矩阵的表示中,$S$的三个分量的不为零的矩阵元由熟悉的表达式给出
\begin{eqnarray} \label{eq:4}
&&\bra{n}S_x\ket{n+1}=\bra{n+1}S_x\ket{n}=\tfrac{1}{2}(n+1)^{\frac{1}{2}}(2S-n)^{\frac{1}{2}};\nonumber\\
&&\bra{n}S_y\ket{n+1}=\bra{n+1}S_y\ket{n}^*=-\tfrac{1}{2}i(n+1)^{\frac{1}{2}}(2S-n)^{\frac{1}{2}};\\
&&\phantom{***}\bra{n}S_z\ket{n}=S-n.\nonumber
\end{eqnarray}
$\mathbf{S}_i$分量的矩阵相对于$n_i$具有\eqref{eq:4}式的形式,但相对于晶体中所有其他原子的$n_j$是对角的。因此,矩阵$\mathbf{S}_i$是公式\eqref{eq:4}和$N-1$个单位矩阵的乘积。

在绝对零度时的平衡态,铁磁晶体的所有自旋平行于外场,所以没有自旋偏差并且所有的$n_i$都消失了。众所周知,如果在这种状态下引入一个自旋偏差到一个特定的原子上,则该自旋偏差不会局限在那个原子上:由于与周围原子的交换相互作用,它将自发地通过晶格传播,从而构成一个“自旋波”,或者更确切地说是一个自旋波包。一个自旋波对应于一个具有确定波矢的自旋偏差的传播。正如我们后面展示的,只要在整个晶格中只有一个自旋偏差,与这个确定波矢的自旋波激发相对应的自旋系统的状态正是哈密顿量\eqref{eq:1}的本征态。总之,如果我们使用撇上标去表记整个晶体的自旋,即矢量求和每个原子的自旋$\mathbf{S}'=\sum_i\mathbf{S}_i$,则\eqref{eq:1}式的本征值可以就被计算:$S_z'=NS$和$S_z'=NS-1$,其中$N$是晶体中的原子总数。

整个自旋波近似的本质在于,对于较小的总晶体自旋偏差,可以将$S_z'=NS-n$状态与完全平行的$S_z'=NS$状态之间的能量差视为大约等于$n$个严格计算得出的单位偏差之和。
换句话说,可加性的假设是与自旋反转有关的。实际上,这种假设是不准确的。例如,如果存在两个自旋偏差,则自旋系统的能量将取决于自旋偏差是否位于(i)同一原子上(仅对于$S>\frac{1}{2}$才可能存在),(ii)相邻原子上,或者(iii)相距更远的原子上(它们的自旋在哈密顿量\eqref{eq:1}中没有耦合)。由于总能量对两个自旋偏差之间距离的依赖性,以及自旋波之间的相互作用,因此不能精确地满足叠加原理。存在两种不同的错误原因。如果我们尝试可视化由此产生的能量修正,则一种修正可以被解释为排斥作用,另一种可以被解释为自旋偏差之间的吸引作用。例如,对于$S=\frac{1}{2}$,排斥是由于给定的某个原子上的自旋偏差不能超过一个。因此,如果两个自旋偏差接近相同的某个晶格位置,由于这种“相互作用”,它们将被散射开。如果$S>\frac{1}{2}$,则仅当存在大于$2S$的自旋波时才会出现这种排斥,然后情况会变得更加复杂。由于两个自旋偏差位于最近邻原子上时的总交换能比位于两个距离更远原子上时的低,因此产生了相互之间的吸引力。在一维晶格中,这种吸引力会导致“自旋复合体”(参见Bethe),即自旋偏差相互绑定的状态,当然也与散射相关联。但是,应该清楚地理解,总的散射截面和由此产生的能量修正由排斥作用和吸引相互作用共同决定。

自旋波近似的现实基础是,在足够低的温度下,自旋偏差之间相互作用的影响可以被忽略。在低温下,磁化强度稍有偏离饱和磁化强度。与晶体中原子的总数相比,晶体中自旋偏差数的平均值较小,因此它们之间的相互作用可以忽略不计。



%%===========================================================
%%===========================================================

\section{铁磁简谐近似} \label{sec:3}

自旋矩阵\eqref{eq:4}与众所周知的简谐振子坐标和动量矩阵之间的相似性是当前处理自旋波方法的起点。本章解释了这种方法如何被用于没有偶极修正\eqref{eq:2}的铁磁性情况。反铁磁性的情况将在第\ref{sec:7}-\ref{eq:8}章中进行讨论。

考虑一个质量为$m$角频率为$\omega$的线性谐振子。这个谐振子的坐标$x$和动量$p$中不为零的矩阵元可以写作
\begin{equation} \label{eq:5}
\begin{array}{l}
\bra{n}x\ket{n+1}=\bra{n+1}x\ket{n}=(\hslash/2m\omega)^{\frac{1}{2}}(n+1)^{\frac{1}{2}},\\
\bra{n}p\ket{n+1}=\bra{n+1}p\ket{n}^*=-i(\hslash m\omega/2)^{\frac{1}{2}}(n+1)^{\frac{1}{2}},
\end{array}
\end{equation}
其中$n$现在对应于谐振子激发的量子数,即,
\begin{equation} \label{eq:6}
\left<n\left|\frac{1}{2m}p^2+\frac{1}{2}m\omega^2x^2-\frac{1}{2}\hslash\omega\right|n\right>=n\cdot\hslash\omega.
\end{equation}
当我们引入两个无量纲的变量
\begin{equation} \label{eq:7}
Q=(m\omega/\hslash)^{\frac{1}{2}}x\text{~~和~~}P=(\hslash m\omega)^{-\frac{1}{2}}p,
\end{equation}
我们可以把\eqref{eq:5}式和\eqref{eq:6}式改写成下面的形式:
\begin{eqnarray} \label{eq:8}
&&\bra{n}S^{\frac{1}{2}}Q\ket{n+1}=\bra{n+1}S^{\frac{1}{2}}Q\ket{n}=\tfrac{1}{2}(n+1)^{\frac{1}{2}}(2S)^{\frac{1}{2}};\nonumber\\
&&\bra{n}S^{\frac{1}{2}}P\ket{n+1}=\bra{n+1}S^{\frac{1}{2}}P\ket{n}^*=-\tfrac{1}{2}i(n+1)^{\frac{1}{2}}(2S)^{\frac{1}{2}};\\
&&\bra{n}S-\tfrac{1}{2}(P^2+Q^2-1)\ket{n}=S-n.\nonumber
\end{eqnarray}

将这些表达式与\eqref{eq:4}式中自旋分量的矩阵元相比较,我们看到两组矩阵之间有惊人的相似性。两组矩阵仅在对角线上或沿对角线的方向具有不消失的矩阵元,并且这些矩阵元的值密切相关。然而,还是有两个重要的区别,即:

(i)矩阵\eqref{eq:8}是无限维度的,而\eqref{eq:4}式中的自旋矩阵只有$(2S+1)$-维。或者换一种说法,自旋矩阵\eqref{eq:4}和矩阵\eqref{eq:8}一样都是无限维度的,但是其中所有$n>2S$的矩阵元都等于$0$而不是等于式\eqref{eq:8}中的值。

(ii)自旋矩阵\eqref{eq:4}中出现的因子$(2S-n)^\frac{1}{2}$在简谐振子矩阵\eqref{eq:8}中被$(2S)^\frac{1}{2}$替换。我们注意到,对于任何$S$值,式\eqref{eq:4}中连接前两个态$n=0$和$n=1$的矩阵元和\eqref{eq:8}式中的相等。尽管相比于$2S$,$n$很小,但当$n$取较大值时对应的矩阵元大约都相等。

现在我们用简谐振子矩阵\eqref{eq:8}来代替\eqref{eq:1}式的自旋系统哈密顿量中的自旋矩阵\eqref{eq:4},也就是说,让我们来做以下替换:
\begin{equation} \label{eq:9}
S_x=S^{\frac{1}{2}}Q;~~S_y=S^{\frac{1}{2}}P;~~S_z=S-\tfrac{1}{2}(P^2+Q^2-1).
\end{equation}
我们就得到了以下哈密顿量表达式:
\begin{eqnarray} \label{eq:10}
&\mathcal{H}=-Hg\beta\sum\nolimits_i[S-\tfrac{1}{2}(P_i^2+Q_i^2-1)]\nonumber\\
&\phantom{~~~~}-2JS\sum\nolimits_{\mathit{nei}}(P_iP_j+Q_iQ_j)+\nonumber\\
&\phantom{~~~~~~~}-2J\sum\nolimits_{nei}[S-\tfrac{1}{2}(P_i^2+Q_i^2-1)]\nonumber\\
&\phantom{~~~~~~~~~~~~~~~~~~~~~~~~~~~~~~~~~~~~~~~~~~~~~~~~~~~~}\times[S-\tfrac{1}{2}(P_j^2+Q_j^2-1)].
\end{eqnarray}

尽管温度很低以至于每个谐振子的$n_i\geqslant 2$态并没有被适当的激发,\eqref{eq:10}式中的哈密顿量描述的耦合简谐振子系统的性质和\eqref{eq:1}式中的自旋系统的性质是相同的。然而在更高温度时,偏差就出现了。对于$S=\frac{1}{2}$,当适当地激发$n\geqslant 2$的能级时,忽略排斥相互作用是不合理的,而对于较大的$S$值,会产生附加修正,因为在较高的温度下,$(2S-n)^\frac{1}{2}$因子可能不再由$(2S)^\frac{1}{2}$代替。谐振子系统\eqref{eq:8}的性质与自旋系统\eqref{eq:1}性质之间的偏差是自旋变量“非谐性”的结果,即自旋矩阵\eqref{eq:4}与简谐振子矩阵\eqref{eq:8}之间的偏差导致的。

可以通过在$P$和$Q$中添加高阶(非协性)项来改进\eqref{eq:9}式的近似,从而不仅使对应于$n=0$和$n=1$的矩阵元一致,而且使对应于$n=2$的矩阵元一致等等,\eqref{eq:9}式中的每个附加项都考虑到了与下一个较高$n$值相对应的下一组矩阵元。我们将说明$S=\frac{1}{2}$时的过程。当我们引入量$S^\pm=S_x\pm iS_y,~a=2^{-\frac{1}{2}}(Q+iP)$和$a^*=2^{-\frac{1}{2}}(Q-iP)$时,我们可以写出\eqref{eq:9}式一阶近似的形式
\begin{equation} \label{eq:11}
S^+=(2S)^{\frac{1}{2}}a;~~S^-=(2S)^{\frac{1}{2}}a^*;~~S_z=S-a^*a.
\end{equation}
\eqref{eq:11}式中的右手边的前四个矩阵元(在态$n=0$和$n=1$之间的那些元素)等于左手边的对应矩阵元。在下一级近似中,我们可以替换\eqref{eq:11}式来使对应于态$n=0,~1,$和$2$的九个矩阵元相一致
\begin{equation} \label{eq:12}
\begin{array}{c}
S^+=(2S)^{\frac{1}{2}}(a-a^*aa);~~S^-=(2S)^{\frac{1}{2}}(a^*-a^*a^*a);\\
S_z=S-a^*a+a^*a^*aa;
\end{array}
\end{equation}
我们可以通过写出\eqref{eq:12}中的矩阵来很简单的验证。对于$S>\frac{1}{2}$,可以以类似的方式考虑出现在$n\leqslant 2S$处的因子$(1-n/2S)^\frac{1}{2}$和现在出现在$n>2S$处的斥力。在文献中,已经报道了许多尝试通过展开平方根$(1-n/2S)^\frac{1}{2}$为$n/2S$的级数并且仅保留前几项来改进\eqref{eq:11}式中的近似,但是忽略了$n>2S$时自旋矩阵\eqref{eq:4}的消失所对应的“排斥”相互作用。

这样,对应于$n\leqslant 2S$的所有矩阵元都被部分修正(这仅对$S>\frac{1}{2}$有意义),而在本方法中(也适用于$S=\frac{1}{2}$)逐次近似\eqref{eq:11},\eqref{eq:12}等等,使连续的矩阵元具有正确的值\eqref{eq:4}。但是应注意,不确定该近似过程是否收敛。由于我们目前仅对常规自旋波近似感兴趣,因此在此不再讨论这一点。

通常的自旋波近似不仅忽略自旋变量的非谐性(即使用一阶近似\eqref{eq:9})而且还忽略自旋偏差间的附加吸引作用。这种相互作用是由哈密顿量\eqref{eq:10}中最后一项中的四次项表示的。在$J>0$时公式\eqref{eq:10}中的四次方项前是负号的事实,在物理上可以被解释为一种自旋偏差间的相互吸引作用。当这些项被忽略时,\eqref{eq:10}式中的哈密顿量变成
\begin{eqnarray} \label{eq:13}
&&\mathcal{H}=E_0'+g\beta(H+H_E)\sum\nolimits_i\tfrac{1}{2}(P_i^2+Q_i^2-1)+\nonumber\\
&&\phantom{~~~~~~~~~~~~~~~~~~~~~~~~~~~~~~~~~~~~}-2JS\sum\nolimits_{\mathit{nei}}(P_iP_j+Q_iQ_j),
\end{eqnarray}
其中
\begin{equation*}
H_E=2JSz/g\beta.
\end{equation*}
这里,并且贯穿全文,$z$代表给定原子的最近邻原子数。$H_E$是我们熟悉的饱和分子场,上式中的常数为
\begin{equation} \label{eq:14}
E_0'=-NHg\beta S-\tfrac{1}{2}Nz\cdot 2JS^2,
\end{equation}
是所有自旋平行于外场的完全饱和状态下的能量。

从\eqref{eq:13}式中我们可以看到,公式\eqref{eq:1}中的哈密顿量现在简化为$N$个由(简化的)坐标和动量 $Q_i$和$P_i$描述的耦合简谐振子系统的哈密顿量。除了\eqref{eq:13}式的最后一项表示相邻协振子之间的耦合(仔细地将其与相邻自旋偏差之间的相互作用加以区别),表达式\eqref{eq:13}的形式正如我们现在将要展示的那样,该表达正是我们期望的形式。因为我们忽略了自旋偏差间的相互作用,我们可能假设每个自旋偏差都被没有自旋偏差的原子包围。那么在一个特定原子上有一个自旋偏差的状态与在那个原子上没有自旋偏差的相应状态之间的能量差就由下式给出:
\begin{eqnarray} \label{eq:15}
&&\Delta E=[-Hg\beta(S-1)-2JzS(S-1)]\nonumber\\
&&\phantom{~~~~~~~~~~~~~~~~~~~}-[-Hg\beta S-2JzS^2]=g\beta(H+H_E),
\end{eqnarray}
如果我们忽略了对应于\eqref{eq:13}式中的耦合项的自旋$x$和$y$分量的影响。当我们想要用一个简谐振子代替自旋,我们必须要求简谐振子的第一激发能等于引入一个孤立的自旋偏差所需要的能量,即必须使谐振子的$\hslash\omega$等于\eqref{eq:15}式中的能量。再次忽略临近简谐振子之间的耦合,我们可以得到哈密顿量
\begin{equation} \label{eq:16}
\mathcal{H}_0=\sum_i(\frac{1}{2m}p_i^2+\frac{1}{2}m\omega^2x_i^2-\frac{1}{2}\hslash\omega)+E_0',
\end{equation}
其中$E'_0$由\eqref{eq:14}式给出并且等于所有谐振子处于基态时的能量。当我们根据\eqref{eq:7}式调整坐标和动量,并使$\hslash\omega$等于\eqref{eq:15}式中的能量时,表达式\eqref{eq:16}与\eqref{eq:13}相同,除了耦合项,即涉及对临近原子的求和项,消失了。在伊辛模型中,该项毫无疑问地被忽略,但是对于本文或自旋波的任何其他处理而言,这都是非常重要的。

寻找耦合谐振子系统的简正模式的问题是一个简单的问题,所需的变换是众所周知的斯拉特(Slater)变换,无论是对于行波
\begin{equation} \label{eq:17}
\begin{array}{l}
Q_k=(1/N)^{\frac{1}{2}}\sum\nolimits_{i}\exp(i\mathbf{k}\cdot\mathbf{r}_i)Q_i;\\
P_k=(1/N)^{\frac{1}{2}}\sum\nolimits_{i}\exp(-i\mathbf{k}\cdot\mathbf{r}_i)P_i;
\end{array}
\end{equation}
或者真正的驻波
\begin{subequations} \label{eq:18}
\begin{equation}
k_x \geqslant 0 \left\{
\begin{array}{l}
Q_k=(2/N)^{\frac{1}{2}}\sum\nolimits_{i}\cos(\mathbf{k}\cdot\mathbf{r}_i)Q_i;\\
P_k=(2/N)^{\frac{1}{2}}\sum\nolimits_{i}\sin(\mathbf{k}\cdot\mathbf{r}_i)P_i;
\end{array}
\right.
\end{equation}
\begin{equation}
k_x < 0 \left\{
\begin{array}{l}
Q_k=(2/N)^{\frac{1}{2}}\sum\nolimits_{i}\sin(\mathbf{k}\cdot\mathbf{r}_i)Q_i;\\
P_k=(2/N)^{\frac{1}{2}}\sum\nolimits_{i}\cos(\mathbf{k}\cdot\mathbf{r}_i)P_i.
\end{array}
\right.
\end{equation}
\end{subequations}
$\mathbf{k}$值的选择需要一些说明。通常会引入所谓的周期性或冯·卡门边界条件,这意味着波函数在穿过晶格后会恢复为其原始值。恰如其分地表达是,这种边界条件的施加意味着在晶体最边缘的自旋仍有临近原子,而不因为处在一侧边缘就没有临近原子。做出这种不真实的假设是为了将所有自旋情况放在同一个水平上,从而使该问题具有充分的周期性,从而简化了分析过程。只要晶体很大,使用这种边界条件所涉及的误差就可以忽略不计,除非存在长程作用力。(然而偶极相互作用可能会导致麻烦,因为它仅以$1/r^3$衰减)。在周期性边界条件下,矢量$\mathbf{k}/2\pi$的允许值通常由晶体所谓的倒格子中的格点表示。对于简单立方晶格和边长为$L$的立方晶体,$k_x,~k_y,$和$k_z$的允许值是$2\pi/L$的整数倍。

就驻波幅度\eqref{eq:18}而言,公式\eqref{eq:13}中的哈密顿量有如下形式
\begin{equation} \label{eq:19}
\mathcal{H}=\sum\nolimits_k\tfrac{1}{2}(P_k^2+Q_k^2-1)\hslash\omega_k+E_0',
\end{equation}
其中
\begin{equation} \label{eq:20}
\hslash\omega_k=Hg\beta+2JS[z-\sum\nolimits_a\cos(\mathbf{k}\cdot\mathbf{a})],
\end{equation} 
求和遍及$z$个矢量$\mathbf{a}$,矢量$\mathbf{a}$连接某个原子核其$z$个临近原子。

根据式\eqref{eq:18}和\eqref{eq:7}中的定义,很容易就可以证明算符的本征值为
\begin{equation*}
\mathfrak{n}_k=\tfrac{1}{2}(P_k^2+Q_k^2-1),
\end{equation*}
是$0,~1,~2,~\cdots$,所以式\eqref{eq:19}中的哈密顿量的本征值由下式给出
\begin{equation} \label{eq:21}
E(\mathfrak{n}_k)=\sum\nolimits_k\mathfrak{n}_k\cdot\hslash\omega_k+E_0',
\end{equation}
其中,简正模式$\mathbf{k}$的频率由式\eqref{eq:20}的色散关系给出。我们使用德语字母$\mathfrak{n}_k$来指定与晶体的各种自旋波相关的量子数,以便将它们与自旋偏差算符或单个原子的量子数区分,我们用$n_i$表示,当包括原子间的交换耦合时,它们不是好的量子数。$\mathfrak{n}_k$或$n_i$每增加一个单位,晶体在$z$方向上的总自旋便减少一个单位的,而$\sum_k\mathfrak{n}_k$或$\sum_i n_i$都可以很好地给出晶体的总旋转偏差。


公式\eqref{eq:20}和\eqref{eq:21}中的结果是众所周知的自旋波近似下自旋系统\eqref{eq:1}低能级的表达式。可用于求出低温下铁磁自旋系统的配分函数。然而,只有在温度很低以至于可以用$k$的级数展开色散关系\eqref{eq:20}的右边部分并且仅保留二次项时,该计算才是可行的。如果我们这样做,式\eqref{eq:20}变成
\begin{equation} \label{eq:22}
\hslash\omega_k=Hg\beta+JS\sum\nolimits_aa^2k^2\cos^2\theta_{k,a},
\end{equation}
其中$\theta_{k,a}$是矢量$\mathbf{a}$和$\mathbf{k}$间的夹角。对于任何立方阵列,$\cos^2\theta_{k,a}$可以用其平均值$\frac{1}{3}$代替。此外,对于一个简单、体心或者面心晶格,我们有$za^2=6l^2$,其中$l$是初基原胞的长度,所以
\begin{equation} \label{eq:23}
\hslash\omega_k=Hg\beta+2JSk^2l^2.
\end{equation}



%%===========================================================
%%===========================================================

\section{自旋波近似下铁磁体的磁化强度和比热} \label{sec:4}

通常通过设置配分函数,并注意自旋波基本上对应于波色-爱因斯坦统计的各种激发可能性,从式\eqref{eq:22}获得磁化强度对温度的依赖性。但是,我们可以通过深入使用简谐振子的形式更简单地获得结果。一个角频率为$\omega_k$的简谐振子的能级,除了基态时的半量子数,对于连续的量子数$\mathfrak{n}_k=1,~2,~\cdots$分别为$\hslash\omega_k,~2\hslash\omega_k,~\cdots$。众所周知的相应平均能量是
\begin{equation} \label{eq:24}
\bar{E}_k=\frac{\hslash\omega_k}{\exp(\hslash\omega_k/kT)-1}.
\end{equation}
没有自旋波,对$\mathfrak{n}_k=1,~2,~\cdots$自旋偏差数分别是$1,~2,~\cdots$。因此谐振子$\omega_k$的平均能量和对应的平均自旋偏差仅仅差了一个因子$\hslash\omega_k$。则不同自旋波的总自旋偏差为
\begin{equation} \label{eq:25}
\braket{NS-S_z'}_{A_V}=\sum\nolimits_k[\exp(\hslash\omega_k/kt)-1]^{-1},
\end{equation}
其中求和遍及$\mathbf{k}$-晶格中所有允许的格点。在$T=0$时总磁矩在$z$方向上$M_z$对其最大值$M_{z0}$的偏离和平均自旋偏差差一个因子$g\beta$。

与交换耦合相比,外场的影响通常较小,因此我们可以采用$\hslash\omega_k=2JSk^2l^2$。如果我们假设晶格是立方体,并且用积分来代替求和,我们可以得到
\begin{eqnarray} \label{eq:26}
&&M_{z0}-M_{z}=g\beta\left(\frac{L}{2\pi}\right)^3\nonumber\\
&&\phantom{~~~~~~~}\times\iiint\frac{\mathrm{d}k_x\mathrm{d}k_y\mathrm{d}k_z}{\exp(2JS(k_x^2+k_y^2+k_z^2)l^2/kT)-1},
\end{eqnarray}
其中$L$是立方晶体的边长,所以$V=L^3$是晶体的总体积,并且在$\mathbf{k}$中间中对整个第一布里渊区积分。在低温时积分可能可以扩展到对整个$\mathbf{k}$空间积分,我们可以得到
\begin{equation} \label{eq:27}
\begin{array}{l}
M_{z0}-M_z\\
\phantom{~~~}=g\beta\displaystyle\left(\frac{L}{2\pi}\right)^3 4\pi\int_0^\infty\frac{k_0^2\mathrm{k_0}}{\exp(2JSk_0^2l^2/kT)-1}\\
\phantom{~~~}=\displaystyle\left(\frac{L}{l}\right)^3\frac{g\beta}{2\pi^2}\left(\frac{kT}{2JS}\right)^{\frac{3}{2}}\int_0^\infty\frac{V^2}{e^{V^2}-1}\mathrm{d}V\\
\phantom{~~~}=\displaystyle\left(\frac{L}{l}\right)^3\frac{g\beta}{2\pi^2}\left(\frac{kT}{2JS}\right)^{\frac{3}{2}}\int_0^\infty V^2(e^{-V^2}+e^{-2V^2}+\cdots)\mathrm{d}V\\
\phantom{~~~}=\displaystyle\left(\frac{L}{l}\right)^3\frac{g\beta}{2\pi^2}\left(\frac{kT}{2JS}\right)^{\frac{3}{2}}\frac{\pi^{\frac{1}{2}}}{4}\left(1+\frac{1}{2^{\frac{3}{2}}}+\frac{1}{3^{\frac{3}{2}}}+\cdots\right).
\end{array}
\end{equation}
在$T=0$时的饱和磁矩为$M_{z0}=g\beta NS$。对于简单立方晶格$N=(L/l)^3$,所以$M_z$可以被写成下面的形式
\begin{equation} \label{eq:28}
\begin{array}{l}
M_z=M_{z0}\displaystyle\left[1-\left(\frac{kT}{2JS}\right)^{\frac{3}{2}}\left(\frac{1}{4\pi}\right)^{\frac{3}{2}}\frac{1}{S}\zeta(\frac{3}{2})\right]\\
\phantom{~~~~}=\displaystyle M_{z0}\left[1-0.1187\left(\frac{kT}{2JS}\right)^{\frac{3}{2}}\frac{1}{S}\right].
\end{array}
\end{equation}
这里和其他地方的$\zeta(m)$代表黎曼zeta函数$\sum_n n^{-m}$。对于体心和面心晶格,正如式\eqref{eq:28}中所示,$T^{\frac{3}{2}}$项的系数分别是二分之一和四分之一,对应于$(L/l)^3$是$\tfrac{1}{2}N$和$\tfrac{1}{4}N$而不是$N$的事实。

公式\eqref{eq:28}就是著名的布洛赫(Bloch)结果,即在$T=0$时自发磁矩对其最大值的偏离与$T^\frac{3}{2}$成正比。布洛赫仅考虑了特殊情况$S=\tfrac{1}{2}$。对于任意$S$值的拓展最早由摩勒(M\"oller)给出。$T^\frac{3}{2}$定理适用于任何三维晶格,因为它是\eqref{eq:26}式积分的一般结构所得到的结果。

自旋波理论的一个有趣结论是它无法预言任何一维或者二维晶格的铁磁性,无论临近原子数是多少。这和海森堡所谓的高斯近似相反,它用最近邻原子数作为唯一标准,所以可以预言简单立方和二维六方晶格有相似的行为。一维和二维晶格不存在铁磁性,反映了以下事实:在这些情况下,$\mathbf{k}$空间中的最终积分通常是$\int[\exp(ak^2)-1]k^{n-1}\mathrm{d}k$类型,其中$n$是晶格维数。对于$n=1$或者$2$,该积分在原点处发散,然后平均自旋偏差为无限大。这种状态意味着铁磁有序状态没有稳定性,因此如果维数小于三,则铁磁性将不存在。

可以通过类似于磁矩的计算来获得对比热的贡献。由于自旋波状态的能量与其对磁矩偏差的贡献相差一个系数$\hslash\omega_k/g\beta$,因此内能由下式给出
\begin{equation} \label{eq:29}
U=\left(\frac{L}{2\pi}\right)^3 4\pi\int_0^\infty\frac{(2JSk_0^2l^2)k_0^2\mathrm{d}k_0}{\exp(2JSk_0^2l^2/kT)-1},
\end{equation}
而不是式\eqref{eq:27}。按级数展开,就像式\eqref{eq:27}我们可以得到
\begin{eqnarray} \label{eq:30}
&&U=\frac{(L/l)^3(2JS)}{2\pi^2}\left(\frac{kT}{2JS}\right)^{\frac{5}{2}}\nonumber\\
&&\phantom{~~~~~~~~~~~~~~~~~~~~~~~~~~~}\times\int_0^\infty V^4(e^{-V^2}+e^{-2V^2}+\cdots)\mathrm{d}V,
\end{eqnarray}
所以零外场时比热可以由下式给出
\begin{equation} \label{eq:31}
C_V=\mathrm{d}U/\mathrm{d}T=cNk(kT/2JS)^{\frac{3}{2}}
\end{equation}
其中对于简单立方晶格
\begin{equation} \label{eq:32}
c=\tfrac{5}{2}\left(\frac{1}{2\pi^2}\right)\frac{3\pi^{\frac{1}{2}}}{8}\zeta(\tfrac{5}{2})=\frac{15}{8\pi^{\frac{3}{2}}}\zeta(\tfrac{5}{2})=0.113.
\end{equation}
对于体心和面心立方晶格,常数$c$的表达式分别有一个额外的因子$\tfrac{1}{2}$或者$\tfrac{1}{4}$。

交换积分和温度下降率为$C_VM_{z0}/(M_{z0}-M_z)$,对于任何具有立方对称性的三维结构,其值为$0.96Nk/2S$。

要强调的是,我们给出的磁化强度和比热公式是基于标准自旋波理论的近似。几位作者已尝试对它们进行修正,以使其获得第\ref{sec:3}章中所描述的吸引和排斥作用,换句话说就是发展了更为完善的理论。不同作者不同意他们的结论,因此,我们不会试图去研究高阶修正这一困难而有争议的问题。最近一次做此尝试的人是戴森(Dyson)。他得出结论:“整个(他的)研究的实际结果简单来说就是,无相互作用的自旋波的线性布洛赫理论足以满足所有实际目的。”

在下一章中,我们将讨论所得到的修正,这不是由于标准的自旋波理论对海森堡或海特勒-伦敦模型的不足,而是因为该模型的一般化包括了偶极结构力。



%%===========================================================
%%===========================================================

\section{偶极结构相互作用的影响} \label{sec:5}

现在,我们研究如何通过自旋偶极耦合叠加在各向同性交换相互作用上来修正与自旋波相关的本征值。这个问题由荷斯坦和普里马科夫首先提出并解决。它比上一章中考虑的纯各向同性问题要复杂得多。其中一个原因是,偶极能量相对于整个晶体的自旋分量不是对角的,因此$\sum_i S_{iz}$不是运动常数(守恒量)。

当加入偶极结构的相互作用,哈密顿函数变成
\begin{eqnarray} \label{eq:33}
&&\mathcal{H}=-2J\sum\nolimits_{\mathit{nei}}\mathbf{S}_i\cdot\mathbf{S}_j+\sum\nolimits_{j>i}D_{ij}[\mathbf{S}_i\cdot\mathbf{S_j}\nonumber\\
&&\phantom{~~~~~~~~~~~~~~~~~~~~~~~~~~~~~~~~~~~~~~~~~~~~~}-3(\boldsymbol{\alpha}_{ij}\cdot\mathbf{S}_i)(\boldsymbol{\alpha}_{ij}\cdot\mathbf{S}_j)].
\end{eqnarray}
一般而言,现在在$i$和$j$上的求和必须扩展到晶体中的所有成对原子上,而不是仅扩展到最近邻原子上。单位矢量$\boldsymbol{\alpha}_{ij}$的分量是连接原子$i$和$j$的矢量$\mathbf{r}_{ij}$的方向余弦$\alpha_{ij}$,$\beta_{ij}$和$\gamma_{ij}$。正如介绍部分所说,如果偶极相互作用有真正的电磁学源头则系数$D_{ij}$的值为$g^2\beta^2/r_{ij}^3$。对$D_{ij}$的贡献可以来自于各向异性交换作用,所得到的临近原子的$D_{ij}$值比经典电磁学理论给出的大很多。尽管荷斯坦和普里马科夫使用的是$D_{ij}$的经典值,但是扩展到更一般的“赝偶极”情况不会带来特殊的困难,所以我们不会局限$D_{ij}$在经典值。

通过将\eqref{eq:9}式代入\eqref{eq:33}来引入自旋波近似。所得到的的项可以很容易地根据$P_i$,$Q_i$和$n_i$的升幂序分组,其中$n_i$被定义为
\begin{equation} \label{eq:34}
n_i=\tfrac{1}{2}(P_i^2+Q_i^2-1).
\end{equation}
表达式\eqref{eq:34}的物理意义是原子$i$的自旋偏差(参见第\ref{sec:2}章),与$P_i$或$Q_i$相比可以认为是二阶的。当自旋波或者简谐振子近似时,哈密顿函数\eqref{eq:33}变成
\begin{equation} \label{eq:35}
\mathcal{H}=-2J\sum\nolimits_{\mathit{nei}}\mathbf{S}_i\cdot\mathbf{S}_j+D_0+D_1+D_2+D_3+D_4,
\end{equation}
其中
\begin{eqnarray}
&&D_0=S^2\sum\nolimits_{j>i}D_{ij}(1-3\gamma_{ij}^2),\label{eq:36}\\
&&D_1=-6S^{\frac{3}{2}}\sum\nolimits_{j>i}D_{ij}[\alpha_{ij}\gamma_{ij}Q_j+\beta_{ij}\gamma_{ij}P_j],\label{eq:37}\\
&&D_2=S\sum\nolimits_{j>i}D_{ij}[(1-\alpha_{ij}^2)Q_iQ_j-3\alpha_{ij}\beta_{ij}(Q_iP_j+P_iQ_j)\nonumber\\
&&\phantom{~~~~~~~~~~~~~~~~~~~~~~~~}+(1-3\beta_{ij}^2)P_iP_j-(1-3\gamma_{ij}^2)2n_i],\label{eq:38}\\
&&D_3=3S^{\frac{1}{2}}\sum\nolimits_{j>i}D_{ij}[\alpha_{ij}\gamma_{ij}(Q_i)n_j+n_iQ_j)\nonumber\\
&&\phantom{~~~~~~~~~~~~~~~~~~~~~~~~~~~~~~~~~~~~~~~~~~}+\beta_{ij}\gamma_{ij}(P_in_j+n_iP_j)], \label{eq:39}\\
&&D_4=\sum\nolimits_{j>i}D_{ij}(1-3\gamma_{ij}^2)n_in_j.\label{eq:40}
\end{eqnarray}
其中可以被解释为非谐修正的$D_3$和$D_4$项具有较高的阶数,因此可以忽略。我们现在必须更加仔细的考虑剩下的$D_0$,$D_1$和$D_2$项。

就本征值的计算而言,$D_0$项是可加常数。从物理上讲,它可以被解释为当所有自旋平行于$z$轴时产生的总偶极或赝偶极能量。但是这是仅在无穷大磁场中才能达到的理想条件,因为偶极耦合本身会破坏磁矩的恒定性,所以会抑制完美的平行。如果偶极相互作用是经典类型,则常数$D_0$可以用经典退磁因子$N_z$表示,该因子被定义为
\begin{equation} \label{eq:41}
N_z=(V/N)\sum\nolimits_j\boldsymbol{r}_{ij}^{-3}(1-3\gamma_{ij}^2)+(4\pi/3)
,\end{equation}
$N_x$和$N_y$也有相似的定义。这里$V$是晶体体积,$N/V$是单位体积的自旋数。只有样品中的退磁场是均匀时退磁因子才有意义,这仅在外场沿其中一个轴,并且椭圆形切割样品的情况下才成立,这正是我们的假设。公式\eqref{eq:41}中对$j$的求和与$i$无关。当我们引入饱和磁化强度
\begin{equation} \label{eq:42}
M_0=(N/V)g\beta S,
\end{equation}
并且利用式\eqref{eq:41},我们可以把式\eqref{eq:36}中的纯经典部分写成下面的形式
\begin{equation} \label{eq:43}
D_0=-\tfrac{1}{2}VM_0[(4\pi/3)M_0-N_zM_0].
\end{equation}
括号中的部分等于洛伦兹场$(4/3)M_0$和退磁场$-N_zM_0$之和,并且因此等于所谓的有效场,即一个格点处的场强。从式\eqref{eq:43}可以看出$D_0$代表有效场中磁矩的能量。因子$\tfrac{1}{2}$的出现是因为有效场是磁矩本身的结果,所以式\eqref{eq:43}是一种自能。式\eqref{eq:43}的结果证明了退磁因子表达式\eqref{eq:41}的正确性。

由式\eqref{eq:37}给出,并且与$Q_i$和$P_i$呈线性的$D_1$项,可以出于所有实际目的而忽略不计,因为通过引入行波\eqref{eq:17},可以表明$D_1$的量级比$D_0$和$D_2$小一个因子$N^\frac{1}{2}$。通常$D_1$会更小,因子为$N^\frac{1}{3}$,即与靠近晶体表面的原子数成正比。如果每个原子都有对称面,这个结论对短程力就是成立的。对于
\begin{equation} \label{eq:44}
\sum\nolimits_jD_{ij}\alpha_{ij}\gamma_{ij}=\sum\nolimits_jD_{ij}\beta_{ij}\gamma_{ij}=0,
\end{equation}
除了靠近晶体表面的原子都成立。对于长程磁性作用力,\eqref{eq:44}式只有在晶体被合适切割时才成立。

二次项$D_2$是偶极项中最重要的。它不能被像\eqref{eq:44}中的关系简化。我们现在的任务就是决定下式的本征值
\begin{equation} \label{eq:45}
-2J\sum\nolimits_{\mathit{nei}}\mathbf{S}_i\cdot\mathbf{S}_j+D_2.
\end{equation}
第一步就是在$D_2$的表达式\eqref{eq:38}中引入行波\eqref{eq:17}。$D_2$中的第一项就变成了
\begin{eqnarray} \label{eq:46}
&&\tfrac{1}{2}N^{-1}S\sum\nolimits_{i,j}D_{ij}(1-3\alpha_{ij}^2)\sum\nolimits_{k,k'}Q_kQ_{k'}\nonumber\\
&&\phantom{~~~~~~~~~~~~~~~~~~~~~~~~~~~~~~~~~~~~~~~}\times\exp[-i\mathbf{k}\cdot\mathrm{r}_i-i\mathbf{k}'\cdot\mathbf{r}_j],
\end{eqnarray}
其中的求和遍及所有的$i$和$j$而非$i<j$的情况。表达式\eqref{eq:46}可以被改写成
\begin{eqnarray} \label{eq:47}
&&\tfrac{1}{2}N^{-1}S\sum\nolimits_{k,k'}Q_kQ_{k'}\sum\nolimits_{i}\exp[-i(\mathbf{k}+\mathbf{k}')\cdot\mathbf{r}_i]\nonumber\\
&&\phantom{~~~~~~~~~~~~~~~~}\times\sum\nolimits_jD_{ij}(1-3\alpha_{ij}^2)\exp[-i\mathbf{k}'\cdot(\mathbf{r}_j-\mathbf{r}_i)].
\end{eqnarray}
我们现在引入下面的晶格求和
\begin{equation} \label{eq:48}
\begin{array}{l}
A_{xx}(\mathbf{k})=S\sum\nolimits_jD_{ij}(1-3\alpha_{ij}^2)\exp[i\mathbf{k}\cdot(\mathbf{r}_j-\mathbf{r}_i)];\\
A_{xy}(\mathbf{k})=-3S\sum\nolimits_jD_{ij}\alpha_{ij}\beta_{ij}\exp[i\mathbf{k}\cdot(\mathbf{r}_j-\mathbf{r}_i)];
\end{array}
\end{equation}
对于$A_{yy}(\mathbf{k})$,$A_{zz}(\mathbf{k})$,$A_{xx}(\mathbf{k})$和$A_{yz}(\mathbf{k})$也是相似的(这些量显然是一个张量的分量),并且我们可以假设这些量与$\mathbf{r}_i$无关。如果忽略边缘效应对于短程力这是合理的。对于长程磁性作用力,情况就更加复杂了。式\eqref{eq:48}中对$D_{ij}$的求和与$r_{ij}^{-3}$成正比,这已经被海勒(Heller)和马库斯(Marcus)与激发理论联系,以及科恩(Cohen)和凯弗(Keffer)讨论过了。对于$\mathbf{k}=0$,式\eqref{eq:48}中的求和与退磁因子\eqref{eq:41}直接关联,并且对于椭球晶体它们与$\mathbf{r}_i$无关。对于$\mathbf{k}\neq 0$,然而,式\eqref{eq:48}中的量仅对离晶体表面距离远大于波长的原子$i$才和$\mathbf{r}_i$无关,因为相位因子迅速变化导致$r_{ij}$相对于波长较大的$j$项的贡献被抵消了。因此,对于与晶体尺寸相比较小的波长,对于大多数原子,对于远程力,求和\eqref{eq:48}与$\mathbf{r}_i$无关。 我们的假设仅在波长与晶体尺寸相当时不成立。如果我们忽略这些效应的影响,式\eqref{eq:47}中对所有$i$的求和可以由下面的性质立刻得到结果
\begin{equation} \label{eq:49}
N^{-1}\sum\nolimits_i\exp(i\mathbf{k}\cdot\mathbf{r}_i)=\delta(\mathbf{k}),
\end{equation}
其中$\mathbf{k}$第一布里渊区中任何允许的倒格矢。表达式\eqref{eq:47}就可以简化为
\begin{equation} \label{eq:50}
\tfrac{1}{2}\sum\nolimits_kA_{xx}(\mathbf{k})Q_kQ_{-k}.
\end{equation}
$D_2$中的其他项可以用同样的方法计算,则最后的$D_2$表达式为
\begin{eqnarray} \label{eq:51}
&&D_2=\tfrac{1}{2}\sum\nolimits_k[A_{xx}(\mathbf{k})-A_{zz}(0)]Q_kQ_{-k}\nonumber\\
&&\phantom{~~~~~~~~~~~~~~~~~~~}+\tfrac{1}{2}\sum\nolimits_k[A_{yy}(\mathbf{k})-A_{zz}(0)]P_kP_{-k}\nonumber\\
&&\phantom{~~~~~~~~~~~~~~~~~~~~~~~~~~~~~~~}+\sum\nolimits_kA_{xy}(\mathbf{k})Q_kP_k+\tfrac{1}{2}NA_{zz}(0).
\end{eqnarray}

对于布拉维式晶体,式\eqref{eq:48}中所有的量都是实数。这时用实驻波\eqref{eq:18}而非行波\eqref{eq:17}就更加方便了。用这些量来表示式\eqref{eq:35}中的哈密顿量(忽略$D_1$,$D_3$和$D_4$项)
\begin{equation} \label{eq:52}
\mathcal{H}=\tfrac{1}{2}\sum\nolimits_k[A(\mathbf{k})Q_k^2+B(\mathbf{k})P_k^2+2C(\mathbf{k})Q_kP_k]+E_0''.
\end{equation}
系数$A$,$B$和$C$由下式给出
\begin{eqnarray} 
&&A(\mathbf{k})=\hslash\omega_k^{(0)}+A_{xx}(\mathbf{k})-A_{zz}(0); \label{eq:53}\\
&&B(\mathbf{k})=\hslash\omega_k^{(0)}+A_{yy}(\mathbf{k})-A_{zz}(0); \label{eq:54}\\
&&C(\mathbf{k})=A_{xy}(\mathbf{k}); \label{eq:55}
\end{eqnarray}
其中$\hslash\omega_k^{(0)}$等于无偶极相互作用下一个波矢为$\mathbf{k}$的自旋波的能量\eqref{eq:20}。常数$E_0''$为
\begin{equation} \label{eq:56}
E_0''=E_0'+\tfrac{1}{2}N(S+1)A_{zz}(0)-\sum\nolimits_k\tfrac{1}{2}\hslash\omega_k^{(0)},
\end{equation}
其中$E_0'$在式\eqref{eq:14}中定义。

通过将方括号中的表达式转换为简单的平方和,可以轻松实现哈密顿量\eqref{eq:52}的对角化。这可以通过正交变换到一组新的变量$Q_k'$和$P_k'$来实现,我们将不写出来。结果为
\begin{equation} \label{eq:57}
\mathcal{H}=\sum\nolimits_k\tfrac{1}{2}(Q_k'^2+P_k'^2)(AB-C^2)^{\frac{1}{2}}+E_0''.
\end{equation}
因为变化是正交的,算符$\tfrac{1}{2}(Q_k'^2+P_k'^2)$的本征值还是$\mathfrak{n}_k+\tfrac{1}{2}$的形式。因此哈密顿量\eqref{eq:57}的本征值可以由下式给出
\begin{equation} \label{eq:58}
E(\mathfrak{n}_k)=\sum\nolimits_k\mathfrak{n}_k\cdot\hslash\omega_k+E_0,~~(\mathfrak{n}=0,1,2,\cdots),
\end{equation}
其中简正模$\mathbf{k}$的频率现在为
\begin{equation} \label{eq:59}
\hslash\omega_k=[A(\mathbf{k})B(\mathbf{k})-C(\mathbf{k})^2]^{\frac{1}{2}},
\end{equation}
并且总零点能等于
\begin{equation} \label{eq:60}
E_0=E_0''+\tfrac{1}{2}\sum\nolimits_k\hslash\omega_k,
\end{equation}
其中$E_0''$由式\eqref{eq:56}给出。

式\eqref{eq:58}的结果表明,在存在偶极相互作用的情况下,自旋系统可以再次由一组简谐振子进行一阶近似描述,其简正模式现在由\eqref{eq:59}给定。式\eqref{eq:59}的结果与第\ref{sec:3}章不同,它涉及一个根号,这反映了哈密顿量\eqref{eq:52}中存在交叉项$Q_kP_k$,如果仅存在各向同性交换耦合则不存在交叉项[参见方程\eqref{eq:19}]。这也可以看作是$S_z'=\sum_i S_{iz}$也不再是运动常数的反映。

现在出现了一个问题:那个特征值相当复杂的公式有什么用?荷斯坦和普里马科夫进行计算主要是为了查看在$T=O$时偶极耦合将饱和磁化强度从对应于完全平行排列的饱和磁化强度降低了多少。结果表明,当用各种常数代替时,通过数值求解求出平均总自旋偏差,由此引起的饱和磁矩的实际减小并不重要,而布洛赫关系$M_0-M=AT^\frac{3}{2}$则不大受到影响。荷斯坦和普里马科夫还研究了固有磁化率$(\partial M/\partial H)$,换句话说,即在易磁化方向已经旋转成平行于所施加磁场的方向之后,磁矩如何受到场强的影响。这个计算非常复杂。他们发现对于大范围场强$(\partial M/\partial H)\propto T/H^\frac{1}{2}$。该结果似乎与有限的可用实验证据一致。

可以借助自旋波研究的另一物理量是接近绝对零度时的铁磁各向异性。这种各向异性可能是由任意$S$的赝偶极耦合引起的,或者是由四极耦合引起的,例如我们在第\ref{sec:1}章中提到的,对于$S\geqslant 1$。在镍中,实验证据表明$S=\tfrac{1}{2}$;因此,这种材料的各向异性为赝偶极子。(纯经典偶极耦合会产生一些各向异性,但数值不足。)在$T=0$时,各向异性来源于零点能\eqref{eq:60}依赖于方向的事实。这种方向依赖性的大小和形式已经由特斯曼(Tessman)给出。通过特定的简化,他的结果和范弗莱克(Van Vleck)在 1937 年基于外斯分子场的半现象学模型得到的结果精确一致。乍一看,结果的一致性似乎非常出色,因为方法不同。但是,在跟踪计算时,由于在任一计算中,在$T=0$时$\Delta S_{zi}=\mp 1$的能量变化都被认为是$\pm 2Jz$($J$是交换积分和$z$是最近邻原子数),因此可以期望达到一致。因此微扰计算中的“频率分母”是相同的,并且当分母与跃迁无关时,完整性关系或矩阵元平方和不变的光谱稳定性定理可确保结果将是 类似的。范佩普(Van Peip)先前还通过相当复杂的计算(主要是自旋波模型和较少激进的假设)检验了偶极模型的铁磁各向异性。结果与特斯曼或范弗莱克获得的结果相差无几。由四极相互作用产生的各向异性的计算要比由偶极子产生的各向异性更简单,因为它可以从一阶微扰计算中推导出来,而不是从二阶微扰计算中得出。通常的理论利用某种分子场模型。然而,也可以用自旋波来解决各向同性交换耦合的问题,可将四极项叠加视为一个很小的微扰作用在无扰动系统上。这种方法已经被凯弗(Keffer)在一篇有趣的论文中使用过了。最惊人的结果是,与铁的实验一致,各向异性应随磁化强度的十次方而变化。

偶极自旋波理论最有意思的一个应用就是铁磁共振,我们将在下一章中讨论。



%%===========================================================
%%===========================================================

\section{铁磁共振吸收的自旋波理论} \label{sec:6}

在铁磁共振吸收的实验中,一个小的振荡磁场$H_x=H_1\cos\omega t$被垂直的施加于一个恒定外场$H$上,假定这个外场沿着$z$方向。正如最早被格里菲斯(Griffith)观测到的那样,从振荡磁场中吸收能量时会发生剧烈的共振。振荡频率的表达式为
\begin{equation} \label{eq:61}
\hslash\omega=g\beta\{[H+(N_x-N_z)M][H+(N_y-N_z)M]\}^{\frac{1}{2}}.
\end{equation}
在这里,$M$是外场为$H$和温度为$T$时晶体的磁化强度。$N_x$,$N_y$和$N_z$是经典的退磁因子,其微观定义在\eqref{eq:41}中给出。简单的表达式\eqref{eq:61}仅对轴沿着$x$,$y$和$z$方向的椭球晶体成立,并且忽略了各向异性。

表达式\eqref{eq:61}最初由基特尔(Kittel)通过经典理论推导出。量子力学方法的推导有范弗莱克(Van Vleck)给出。两种推导都是基于运动方程的推导,并且都没使用自旋波。这篇文章通过自旋波的方法推导出\eqref{eq:61}式,尽管这并不是最简单的方法去推导\eqref{eq:61}式。乍一看,似乎基特尔的结果与自旋波模型之间存在明显的矛盾,因为后者具有$3N$个简谐振子或简正模,其中$N$是晶体中磁性原子的数量。这些合适的振动频率中的大多数是截然不同的,因此我们可能会期望实际上是连续的共振频率,而不是像\eqref{eq:61}式所预测的那样只有一条线。如下所示,对这个问题的答案是,在实际上无数的自旋波频率中,通常只有一个在磁共振吸收中“活跃”,即只有这一个频率是被磁偶极子辐射的选择定则所允许的。该特定频率是在不同原子的所有自旋中对称的频率,对应于零波矢$\mathbf{k}$。实质上,这一点是由波尔德(Polder)给出的式\eqref{eq:61}的推导得出的,我们的目的是通过我们的谐振子模型得出相同的结果。需要注意的是,和一般自旋波频率不同,式\eqref{eq:61}并没有包括交换积分。如果没有退磁修正,式\eqref{eq:61}中的频率将降低到拉莫尔频率$g\beta H$,和第\ref{sec:3}章中$\mathbf{k}=0$时的特殊例子中的自旋波频率相同[参看方程\eqref{eq:20}]。由于总偶极矩$S_z'=\sum_i S_{iz}$在矩阵乘法中与交换作用能$-2J\sum_{\mathit{nei}}\mathbf{S}_i\cdot\mathbf{S}_j$对易,因此在铁磁共振中不会出现交换频率。

为了从自旋波理论推导共振频率,我们假设振荡磁场$H_x$的趋肤深度比晶体的尺寸大。则磁场$H_x$就可以被认为是在晶体中均匀分布的,而自旋系统和振荡磁场的相互作用由下式给出
\begin{equation} \label{eq:62}
\mathcal{H}'=-H_zg\beta\sum\nolimits_iS_{ix},
\end{equation}
其中的求和遍及晶体中的所有原子。在自旋波近似中,算符$S_{ix}$被简谐振子坐标$S^{\frac{1}{2}}Q_i$代替,参见方程\eqref{eq:9},则$\mathcal{H}'$变成
\begin{equation} \label{eq:63}
\mathcal{H}'=-H_xg\beta S^{\frac{1}{2}}\sum\nolimits_iQ_i=-H_xg\beta(NS)^{\frac{1}{2}}Q_0,
\end{equation}
根据方程\eqref{eq:17},其中的$Q_0$是完全对称的正交坐标$N^\frac{1}{2}\sum_i Q_i$。最终,我们必须对存在偶极相互作用时的正交坐标$Q_k'$和$P_k'$进行正交变换,参见方程\eqref{eq:57}。现在$Q_k'$和$P_k'$仅依赖于具有相同波矢$\mathbf{k}$的$Q_k$和$P_k$,因此$Q_0$仅仅是$Q_0'$和$P_0'$的线性方程。最后$\mathcal{H}'$变成下面的形式
\begin{equation} \label{eq:64}
\mathcal{H}'=-H_xg\beta(NS)^{\frac{1}{2}}(aQ_0'+bP_0'),
\end{equation}
其中$a$和$b$是式\eqref{eq:53},\eqref{eq:54}和\eqref{eq:55}中量的函数,它们的精确形式现在是无关紧要的。

自旋系统的能量本征态由量子数$\mathfrak{n}_k$表征,其表示各种简正模式的激发程度,或者等价地表示具有波矢$\mathbf{k}$的激发自旋波的数量。因为$\mathcal{H}'$仅含有零级简正模的变量$Q_0'$和$P_0'$,则$\mathcal{H}'$在$\mathfrak{n}_k$表示下的矩阵元,在态$\mathfrak{n}_0,~\cdots,~\mathfrak{n}_k,~\cdots$和$\mathfrak{n}_0',~\cdots,~\mathfrak{n}_k',~\cdots$中仅当下式成立时不为零
\begin{equation} \label{eq:65}
\mathfrak{n}_0'=\mathfrak{n}_0\pm 1~~\text{和}~~\mathfrak{n}_k'=\mathfrak{n}_k~~\text{对于}~~k\neq 0.
\end{equation}
因此振荡磁场只能激发零级简正模,即零波数的自旋波。这一简正模的频率为
\begin{equation} \label{eq:66}
\hslash\omega_0=[A(0)B(0)-C(0)^2]^{\frac{1}{2}},
\end{equation}
显然等于共振频率。当我们用表达式\eqref{eq:53},\eqref{eq:54}和\eqref{eq:55}代替$A(0)$,$B(0)$和$C(0)$,式\eqref{eq:66}中的共振频率变成
\begin{eqnarray} \label{eq:67}
&&\hslash\omega_0=\{[g\beta H+A_{xx}(0)-A_{zz}(0)]\nonumber\\
&&\phantom{~~~~~~~~~~~~~~~}\times[g\beta H+A_{yy}(0)-A_{zz}(0)]-A_{xy}(0)^2\}^{\frac{1}{2}}.
\end{eqnarray}

如果我们假设仅存在纯经典偶极耦合,表达式\eqref{eq:67}就可以简化为基特尔(Kittel)公式\eqref{eq:61}。根据式\eqref{eq:41}和\eqref{eq:48}我们有
\begin{eqnarray} \label{eq:68}
&&A_{zz}(0)=S\sum\nolimits_jg^2\beta^2r_{ij}^{-3}(1-3\gamma_{ij}^2)\nonumber\\
&&\phantom{~~~~~~~~~~~~~~~~~~~~~~~~~~~~~~~~}=g\beta N_z(N/V)g\beta S=g\beta N_zM_0,
\end{eqnarray}
相似的有$A_{xx}(0)$和$A_{yy}(0)$,而$A_{xy}(0)$可能会消失。当这些结果被代入方程\eqref{eq:67},我们就得到了共振频率
\begin{equation} \label{eq:69}
\hslash\omega_0=g\beta\{[H+(N_x-N_z)M_0][H+(N_y-N_z)M_0]\}^{\frac{1}{2}},
\end{equation}
这和基特尔的表达式\eqref{eq:61}是一致的,除了磁化强度$M$被饱和磁化强度$M_0$代替了。因此,正如被期待的那样,自旋波理论给出了仅在低温下的共振频率值。

现在让我们来考虑短程偶极作用力。如果晶格是立方对称的,这将对共振频率没有影响。所有的量$A_{xx}(0)$等都一样消失了(除了通常被忽略的趋肤效应)因为它们将导致对外场与晶轴的方向余弦的二次依赖,而这种二次依赖与立方对称性不相符合。短程力只能通过我们之前忽略的非谐项$D_3$和$D_4$的高阶微扰来影响共振频率。然而,如果没有立方对称性,情况就完全不同了。短程力对量$A_{xx}(0)$等的贡献没有消失,这些作用力导致的各向异性会对共振频率有一个直接影响。正如范弗莱克(Van Vleck)之前讨论过的那样,自旋间的二次耦合导致的各向异性也有相同的作用。

最终,跟随着路丁格(Luttinger)和基特尔(Kittel),我们讨论了共振频率表达式\eqref{eq:61}的另一种推导方式。从振荡磁场和自旋系统间的相互作用与总自旋$\mathbf{J}=\sum_i\mathbf{S}_i$(我们之前用$\mathbf{S}'$表示)对易的事实来看,振荡磁场只能引起有着相同总自旋量子数$J$的态之间的跃迁。因此,为了计算共振频率,我们必须计算与外磁场和退磁场相互作用的磁化晶体的特定J值相对应的态之间的能量差。则以一个大小不变$M$的磁化强度$\mathbf{M}$的函数描述晶体能量的经典哈密顿量由下式给出
\begin{equation} \label{eq:70}
\mathcal{H}=-VM_zH+\tfrac{1}{2}V(N_xM_x^2+N_yM_y^2+N_zM_z^2).
\end{equation}
路丁格(Luttinger)和基特尔(Kittel)发现为了推导题词共振频率,计算式\eqref{eq:70}的本征值就足够了,而不需要荷斯坦和普里马科夫使用(我们在第\ref{sec:5}中使用过)的那些更复杂的哈密顿函数。路丁格(Luttinger)和基特尔(Kittel)表示,对式\eqref{eq:70}的本征值而言,在一定程度上近似就可以由式\eqref{eq:60}在空间均匀分开。他们的证明使用了一种巧妙,但是某种程度上迂回的方法,一个不同的方程被另一个不同的方程代替,同时量化轴垂直于磁场。而简谐振子一个非常直观的应用,并使用量化轴平行于磁场也可以得到正确的结果。

如果我们假设$M^2=M_x^2+M_y^2+M_z^2$是一个常数,这一假设是合理的因为$M^2$在量子力学中与式\eqref{eq:70}对易,而后,除了一个无趣的附加常数,式\eqref{eq:70}等同于
\begin{equation} \label{eq:71}
\mathcal{H}=-VM_zH+\tfrac{1}{2}V[(N_x-N_z)M_x^2+(N_y-N_z)M_y^2].
\end{equation}
我们忽略饱和效应,即我们假设$M_z$微偏离其最大值$M$。像布隆伯根(Bloembergen)和王(Wang)观察到的那种出现于极强振荡磁场下的饱和效应,我们在此不作考虑。

通过用$g\beta\mathbf{J}/V$代替$\mathbf{M}$得到对应于式\eqref{eq:71}的量子力学算符,
\begin{eqnarray} \label{eq:72}
&&\mathcal{H}=-Hg\beta J_z+\tfrac{1}{2}(g\beta M/J)\nonumber\\
&&\phantom{~~~~~~~~~~~~~~~~~~~~~~~}\times[(N_x-N_z)J_x^2+(N_y-N_z)J_y^2].
\end{eqnarray}
我们对$J$取极大值($10^{20}量级$)时和平均值($J_z$没有明显偏离$J$)时的式\eqref{eq:72}的本征值感兴趣,这是对经典情况$M_z\approx M$的类比。[如果$N_z\neq N_y$,$J_z$不与式\eqref{eq:72}对易并且不对应于一个好的量子数;则我们必须处理期望或者平均值而非$J_z$的本征值。]但这正是简谐振子近似\eqref{eq:7}的条件,即
\begin{equation} \label{eq:73}
J_x=J^{\frac{1}{2}}Q;~~J_y=J^{\frac{1}{2}}P;~~J_z=J-\tfrac{1}{2}(P^2+Q^2-1),
\end{equation}
是一个极好的近似。当我我们将式\eqref{eq:73}代入\eqref{eq:72},我们得到
\begin{equation} \label{eq:74}
\mathcal{H}=\text{const}+g\beta\tfrac{1}{2}(aQ^2+bP^2),
\end{equation}
其中$a=H+(N_x-N_z)M$和$b=H+(N_y-N_z)M$。做一个简单的变换,
\begin{equation} \label{eq:75}
Q=(b/a)^{\frac{1}{2}}Q',~~P=(a/b)^{\frac{1}{2}}P',
\end{equation}
则哈密顿量变成
\begin{equation} \label{eq:76}
\mathcal{H}=\text{const}+g\beta(ab)^{\frac{1}{2}}\tfrac{1}{2}(P'^2+Q'^2),
\end{equation}
可见当$M_z\approx M$时$H$的本征值是均匀地按能量差$g\beta(ab)^\frac{1}{2}$分开。代入$a$和$b$的值,就简化为共振频率的表达式\eqref{eq:61}。

从这一推导和类似的经典推导可以明显看出,对于高能级($J_z$的平均值和$J$差别很大),均匀分隔的\eqref{eq:61}将导致偏差的出现,因为对于这些态而言,简谐振子近似\eqref{eq:73}不再是一个好的近似。

在结束本章时应该提醒注意的是,通过实验越来越清楚表明,铁磁共振现象通常过于复杂,无法用简单的基特尔公式\eqref{eq:61}描述。有时会出现副峰,特别是在高阶时,必须考虑饱和效应。因为偶极作用力仅按三次幂下降,所以不能像通常那样像样地去掉边界修正,并且由于主模和副模的频率几乎相等而产生干扰,因此会产生重要的失真效果。我们在此不讨论这一复杂但是实际上非常重要的问题,有关这一问题的文献正在快速地增加。应特别参考克拉斯顿(Clogston),苏尔(Suhl),沃克(Walker)和安德森(Anderson)的论文。

%%===========================================================
%%===========================================================

\section{反铁磁简谐近似} \label{sec:7}

自旋系统低温性质的自旋波理论的出发点是自旋系统的基态:自旋波被引入是为了描述自旋系统对于基态的微小偏离。当我们想研究一个微小扰动的影响时,比如施加一个弱外场或者热能,自旋波理论只有在微扰下基态仍稳定的情况时才成立。这时,自旋系统对基态的偏离可以用一系列合适的简正模来帮助分析,正如第\ref{sec:3}章中对铁磁自旋系统解释的那样。如果微扰下基态不稳定,自旋波理论就失效了。这时,自旋系统对其初始状态的大偏离就不能用自旋波来描述,这种情况下自旋波理论的结果的是发散的。

让我们先来考虑基态稳定性的问题,自旋波理论对铁磁情况的可适用性在第\ref{sec:3}章已经做了讨论。我们假设不存在外场,并且没有任何各向异性,自旋系统的哈密顿函数为
\begin{equation} \label{eq:77}
\mathcal{H}=-2J\sum\nolimits_{\textit{nei}}\mathbf{S}_i\cdot\mathbf{S}_j.
\end{equation} 
基态时所有的自旋都平行,但是最后的自旋方向是任意的,所以基态是简并的。我们当然可以选择自旋指向某个特定方向的特定状态作为自旋波理论的初始状态。但是对于哈密顿量\eqref{eq:77}这一状态在微扰下是不稳定的。比如,很少的热能,就将会导致总磁矩的消失,对应于无各向异性场和外场的情况下热平衡时总磁矩为零的事实。一个自由旋转的总自旋的存在,将导致自旋波理论的发散,这一理论将无法被使用。有些自旋波($k=0$)将没有能量,或者频率为零,并且在任何一个有限的温度下,这些自旋波模式将以任意大小的量子数被激发。通过消除基态简并,或者假设一开始就有一个足够大的外场存在,再或者引入一个等效各向异性场$H_A$,这一困难就可以被避免。基态时,自旋指向场的方向。在温度为$T$时($kT$与$g\beta H$或者$g\beta H_A$相比是个小量),只有一个对基态很小的偏离,这可以通过自旋波的帮助来描述。事实上,各向异性场来自于自旋间的各向异性耦合,比如说第\ref{sec:1}章中讨论过的赝偶极类型。当这种耦合被包括在\eqref{eq:77}式的哈密顿量中,基态就自动变成非简并的了,除了一个由晶体对称性导致的平凡简并。不幸的是,这一问题太难得到令人满意的处理,因此我们必须使用引入一个等效各向异性场的巧妙方法。铁磁性问题中各向异性并不是一个主要角色,正如第\ref{sec:3}章中一样,我们通常引入一个外场来消除基态简并。


现在让我们来考虑一个反铁磁自旋系统。限制我们的讨论在一个可以被分为子晶格$1$和$2$的简单或者体心立方结构中,通过这种方法,子晶格$1$的一个原子的所有最近邻原子都在子晶格$2$中,反之亦然。这种限制并不是必要的,引入它只是为了尽量简化讨论。如果只有近邻自旋间的各向同性耦合,自旋系统的哈密顿量由下式给出
\begin{equation} \label{eq:78}
\mathcal{H}=2J\sum\nolimits_{\mathit{nei}}\mathbf{S}_i\cdot\mathbf{S}_j,
\end{equation}
其中$i$代表子晶格$1$中的原子,而$j$代表子晶格$2$中的原子;耦合常数$J$为正数。所谓两种子晶格的反铁磁状态,是指在子晶格$1$的自旋方向之间存在一定的长程顺序,在相反方向的子晶格$2$的自旋之间存在相似顺序的状态。完全各向同性时的哈密顿量\eqref{eq:78}对应的态是简并的,因为两个子晶格的反平行磁矩的共同方向是任意的。这种简并不能被外场消除。对于一个不太大的外场,自旋将反铁磁地布置在垂直于外部场的平面中,但是自旋在平面中的方向仍然是任意的。

在任何真正的反铁磁晶体中,这种简并都可以通过各向异性消除,正如我们将看到的,在这里各向异性起着比在铁磁体中更重要的作用。各向异性必须再次被以一种等效各向异性场的形式引入,但是它和铁磁各向异性场是十分不同的。必须确保子晶格$1$中的自旋优先朝向$+z$方向,而子晶格$2$中的自旋优先朝向$-z$方向。这可以通过在子晶格$1$格点上引入一个指向$+z$方向,在子晶格$2$中指向$-z$方向的假想场$H_A$来实现。则哈密顿函数变成
\begin{equation} \label{eq:79}
\mathcal{H}=2J\sum\nolimits_{\mathit{nei}}\mathbf{S}_i\cdot\mathbf{S}_j-H_Ag\beta(\sum\nolimits_iS_{iz}-\sum\nolimits_jS_{jz}),
\end{equation}
其中$i$遍及子晶格$1$,而$j$遍及子晶格$2$,这是我们在这一章中都遵循的一个约定。在无限强各向异性场$H_A$的极限下,可以忽略\eqref{eq:79}式中的耦合项,而且\eqref{eq:79}式的基态是子晶格$1$中的所有自旋都指向$+z$方向以及子晶格$2$上的所有自旋都指向$-z$方向的状态。如果$H_A$是个有限值,只要$g\beta H_A$不比$J/N$小好几个数量级(其中$N$是晶体中的总原子数),则式\eqref{eq:79}的基态仅微弱的依赖于$H_A$的值。更准确地说,$g\beta H_A$必须比式\eqref{eq:78}的第一激发态和基态间$J/N$量级的能量差要大。(这对于一维晶体是不成立的。卡斯特林(Kasteleyn)已经表明,对于线性链,当各向异性变得小于某个有限临界值时,基态是无序的。这对于二维晶体可能也是成立的。)

对于一个有限值的$N$和严格等于零的$H_A$,哈密顿量\eqref{eq:79}中的基态,即各向同性哈密顿量\eqref{eq:78}的基态,是未知的。幸运的是,这种状态没有物理意义,因为它不会出现在真实的反铁磁晶体中。一个有限值各向异性的基态是式\eqref{eq:78}的低级态的确定线性组合,我们就是将式\eqref{eq:78}态的这种确定线性组合称为反铁磁基态。对应于假设的两种子晶格结构,我们相应地定义了反铁磁基态,对于小$H_A$而言,其为式\eqref{eq:79}的极限基态,或者更确切地说,即当第一个$N$趋于无穷大$H_A$趋于零食的状态。如前所述,也正如下面详细展示的那样,这种状态和$H_A$取有限值时的基态并没有很大不同。特别地,它和$H_A$趋于无穷大是的完全有序状态也没有很大不同。这种完全有序状态被使用作为反铁磁自旋波理论的初始状态。引入反铁磁自旋波来描述自旋系统的状态与该完全有序状态的微小偏离,这是由于实际上$H_A$的值是有限的而不是无限的,以及应用了其他一些微扰引起的 例如一个弱外场或少量热能。如果自旋波理论给出了一个收敛的结果,假定的反铁磁序列至少对应于能量的相对最小值。这一结构是否也对应于绝对最小值还不能决定,除非和自旋间存在真正各向异性耦合的自旋系统的问题被解决了。

在完全有序状态下,子晶格$1$中的所有自旋都指向$+z$方向,而子晶格$2$中的所有自旋都指向$-z$方向,即对于所有的$i$和$j$,有$S_{iz}=S$和$S_{jz}=-S$。如果$S_{iz}$不等于$S$,或者$S_{jz}$不等于$-S$,我们就说在原子$i$或者$j$处分别出现了一个自旋偏差。那么我们已经定义的反铁磁基态就只有相对较少的自旋偏差;对于一个简单立方格子,结果平均大概有$93\%$的原子上没有自旋偏差。这种状态因此可以被视作一个较好的自旋波理论的近似。总之,它在微扰下并不稳定,而且如果研究它对一个外场或者热能作用下自旋系统的状态的影响,引入一个有限的各向异性是十分必要的。各向异性在保证稳定性上非常有效;$g\beta H_A$并不一定要比微扰大,只要$g\beta(H_A H_E)^\frac{1}{2}$足够大就行了,其中$H_E$是外斯(Weiss)分子场。这里分子场通过各向异性作用引入,这正是相比于铁磁情况下,各向异性在这里更加重要的原因。典型的取值$H_A=10^2oe,~H_E=10^5oe$,所以一个$10^2oe$数量级的外场仍能视作一个微扰。只要微扰和$H_A$取有限值的事实导致的对完全有序态的偏离很小,自旋波理论可以令人满意地描述该微扰的影响。


我们现在展示如何在反铁磁情况中分析引入简谐振子近似。当我们包含与平行外场$H_z=H$的相互作用,哈密顿量\eqref{eq:79}变成
\begin{eqnarray} \label{eq:80}
&&\mathcal{H}=2J\sum\nolinebreak_{\mathit{nei}}\mathbf{S}_i\cdot\mathbf{S}_j-(H+H_A)g\beta\sum\nolimits_iS_{iz}+\nonumber\\
&&~~~~~~~~~~~~~~~~~~~~~~~~~~~~~~~~~~~~~~~~~~~~~-(H-H_A)g\beta\sum\nolimits_jS_{jz},
\end{eqnarray}
在完全有序状态下我们有$S_{iz}=S$和$S_{jz}=-S$,我们因此引入下面的自旋-偏差量子数
\begin{equation} \label{eq:81}
n_i=S-m_i,~~\text{和}~~n_j=S+m_j.
\end{equation} 
公式\eqref{eq:7}中对自旋变量的简谐振子近似必须被改写成
\begin{eqnarray} 
&&S_{ix}=S^{\frac{1}{2}}Q_i;~~S_{iy}=S^{\frac{1}{2}}P_i;~~S_{iz}=S-\tfrac{1}{2}(P_i^2+Q_i^2-1);\label{eq:82}\\
&&S_{jx}=S^{\frac{1}{2}}Q_j;~~S_{iy}=-S^{\frac{1}{2}}P_j;~~S_{jz}=-S+\tfrac{1}{2}(P_j^2+Q_j^2-1).\label{eq:83}
\end{eqnarray} 
式\eqref{eq:82}和式\eqref{eq:83}中左手边的矩阵元等于$n_i$和$n_j$取$0$和$1$时右手边对应的矩阵元,即$m_i$等于$S$和$S-1$和$m_j$等于$-S$和$-S+1$的情况。当我将式\eqref{eq:82}和\eqref{eq:83}代入式\eqref{eq:80}的哈密顿量,我们可以得到
\begin{eqnarray} \label{eq:84}
&&\mathcal{H}=E_0'+2JS\sum\nolimits_{\mathit{nei}}(Q_iQ_j-P_iP_j)-2J\sum\nolimits_{\mathit{nei}}n_in_j\nonumber\\
&&~~~~~~~~~~~~~~~~~~~~+[2JS_z+g\beta(H_A+H)]\sum\nolimits_in_i\nonumber\\
&&~~~~~~~~~~~~~~~~~~~~~~~~~~~~+[2JS_z+g\beta(H_A=H)]\sum\nolimits_jn_j,
\end{eqnarray}
其中我们引入自旋偏差算符
\begin{equation} \label{eq:85}
n_i=\tfrac{1}{2}(P_i^2+Q_i^2-1)~~\text{和}~~n_j=\tfrac{1}{2}(P_j^2+Q_j^2-1),
\end{equation}	
而且其中的$E_0'$等于
\begin{equation} \label{eq:86}
E_0'=-NzJS^2-H_Ag\beta NS.
\end{equation}
简谐振子系统\eqref{eq:84}的性质和式\eqref{eq:80}的性质是一样的,只要谐振子$n\geqslant 2$的态没有被显著激发。足够大的$H_A$值正是这种情况。然而当$H_A$较小时,式\eqref{eq:84}仍是一个好的近似。

除了式\eqref{eq:82}和\eqref{eq:83}中用谐振子变量代替自旋变量的近似之外,我们还必须忽略由\eqref{eq:84}式中四次项表示的自旋偏差间的吸引相互作用。当我们做了这些简化后,可以通过引入合适的简正模来对角化哈密顿量。我们首先分别引入子晶格$1$和$2$的行波
\begin{equation} \label{eq:87}
\left\{
\begin{array}{l}
Q_{1k}=(2/N)^{\frac{1}{2}}\sum\nolimits_i\exp(i\mathbf{k}\cdot\mathbf{r}_i)Q_i;\\
P_{1k}=(2/N)^{\frac{1}{2}}\sum\nolimits_i\exp(-i\mathbf{k}\cdot\mathbf{r}_i)P_i;
\end{array}
\right.
\end{equation}

\begin{equation} \label{eq:88}
\left\{
\begin{array}{l}
Q_{2k}=(2/N)^{\frac{1}{2}}\sum\nolimits_j\exp(-i\mathbf{k}\cdot\mathbf{r}_j)Q_j;\\
P_{2k}=(2/N)^{\frac{1}{2}}\sum\nolimits_j\exp(i\mathbf{k}\cdot\mathbf{r}_j)P_j.
\end{array}
\right.
\end{equation}
波矢$\mathbf{k}$可以取边界条件允许的$\tfrac{1}{2}N$个不同值,以保证我们有正确的变量数。用这些变量表示哈密顿量得
\begin{eqnarray} \label{eq:89}
&&\mathcal{H}=E_0'+\sum\nolimits_k\{[2JSz+g\beta(H_A+H)]\nonumber\\
&&~~~~~~\times\tfrac{1}{2}(P_{1k}P_{1-k}+Q_{1k}Q_{1-k}-1)\nonumber\\
&&~~~~~~+[2JSz+g\beta(H_A-H)]\tfrac{1}{2}(P_{2k}P_{2-k}+Q_{2k}Q_{2-k}-1)\nonumber\\
&&~~~~~~~~~~~~~~~~~~~~~~~~~~~~~~+2JS\gamma_k(Q_{1k}Q_{2k}-P_{1k}P_{2k})\}.
\end{eqnarray}
其中$\gamma_k$由下式给出
\begin{equation} \label{eq:90}
\gamma_k=z^{-1}\sum\nolimits_a\cos(\mathbf{k}\cdot\mathbf{a}),
\end{equation}
其中求和遍及$z$个连接原子和其$z$个最近邻原子的矢量$\mathbf{a}$。通过一个线性变换(参见参考文献 7 和 40)
\begin{equation} \label{eq:91}
\begin{array}{l}
\left\{
\begin{array}{l}
Q_1k=c_{1k}Q_{1k}'+c_{2k}Q_{2-k}';\\
Q_{2-k}=c_{2k}Q_{1k}'+c_{1k}Q_{2-k}';
\end{array}
\right.\\
\left\{
\begin{array}{l}
P_{1k}=c_{1k}P_{1k}'-c_{2k}P_{2-k}',\\
P_{2-k}=-c_{2k}P_{1k}'+c_{1k}P_{2-k}',
\end{array}
\right.
\end{array}
\end{equation}
其中$c_{1k}=c_{1-k},~c_{2k}=c_{2-k}$和$c_{1k}^2-c_{2k}^2=1$,我们可以消去式\eqref{eq:89}中的交叉项。当我们将式\eqref{eq:91}带入式\eqref{eq:89}的哈密顿量中,我们可以发现交叉项消失了如果系数$c_{ik}$有如下值
\begin{equation} \label{eq:92}
c_{1k}=\rho_k/(\rho_k^2-\gamma_k^2)^\frac{1}{2};~~c_{2k}=-\gamma_k/(\rho_k^2-\gamma_k^2)^\frac{1}{2},
\end{equation}
其中$\gamma_k$由\eqref{eq:90}式给出,而$\rho_k$由下式定义
\begin{equation} \label{eq:93}
\rho_k=1+H_A/H_E+[(1+H_A/H_E)^2-\gamma_k^2]^\frac{1}{2}.
\end{equation}
其中量$H_E$由下式给出
\begin{equation*}
H_E=2JSz/g\beta,
\end{equation*}
这就是完全有序态下的外斯分子场。用撇上标的变量可以表示标准形式的哈密顿量为
\begin{eqnarray} \label{eq:94}
&&\mathcal{H}=E_0+\sum\nolimits_k\tfrac{1}{2}(P_{1k}'P_{1-k}'+Q_{1k}'Q_{1-k}'-1)\hslash\omega_{1k}\nonumber\\
&&~~~~~~~~~~~~~~~~~~~~+\sum\nolimits_k\tfrac{1}{2}(P_{2k}'P_{2-k}'+Q_{2k}'Q_{2-k}'-1)\hslash\omega_{2k}.
\end{eqnarray}
所以本征值由下式给出
\begin{equation} \label{eq:95}
E(\mathfrak{n}_{1k},\mathfrak{n}_{2k})=E_0+\sum\nolimits_k(\mathfrak{n}_{1k}\hslash\omega_{1k}+\mathfrak{n}_{2k}\hslash\omega_{2k}),
\end{equation}
其中$\mathfrak{n}_{1k}$和$\mathfrak{n}_{2k}$取值为$0,~1,~2,~\cdots$。简正自旋波模的频率由下式给出
\begin{equation} \label{eq:96}
\left\{
\begin{array}{l}
\hslash\omega_{1k}=g\beta[(H_E+H_A)^2-\gamma_k^2H_E^2]^\frac{1}{2}+g\beta H;\\
\hslash\omega_{2k}=g\beta[(H_E+H_A)^2-\gamma_k^2H_E^2]^\frac{1}{2}-g\beta H;
\end{array}
\right.
\end{equation}
而基态能量$E_0$等于
\begin{equation} \label{eq:97}
E_0=E_0'+\tfrac{1}{2}\sum\nolimits_k(\hslash\omega_{1k}+\hslash\omega_{2k}),
\end{equation}
其中$E_0'$由\eqref{eq:86}式给出。

正如我们可以从式\eqref{eq:96}中看出的那样,简正模的两个分支在$H=0$时相同。关于这些自旋波模式的详细的讨论和经典解释由凯弗(Keffer),卡普兰(Kaplan)和雅菲特(Yafet)给出。从式\eqref{eq:96}中我们还可以看出,因为$\gamma_k^2\leqslant 1$,模式$2,~0$有最小的激发能,即
\begin{equation} \label{eq:98}
\hslash\omega_{20}=g\beta(H_c-H);~~H_c=[H_A(H_A+2H_E)]^\frac{1}{2}.
\end{equation}
当$H$大于等于临界值$H_c$时,甚至在绝对零度时简谐振子近似都显然失效了,因为模式$2,~0$将被激发到一个任意高的能级。这对应于一个众所周知的事实,当平行场$H$超过临界值$H_c$时,自旋转向垂直于$H$的方向。这导致的自旋系统状态的大幅变化不能用自旋波近似来描述,并且该理论中出现了发散。但是至少在绝对零度时,临界场的值由式\eqref{eq:98}正确给出。

垂直外场的情况可以用相似的方法处理(参见,比如说库博(Kubo)的文章)。式\eqref{eq:96}和\eqref{eq:97}中的结果和相应的垂直外场情况的结果可以被用于计算平行和垂直磁化系数。反铁磁体的比热将在第\ref{sec:9}章与铁磁体相联系一起讨论。

至于反铁磁共振吸收,在第\ref{sec:6}章中铁磁情况下振荡磁场只能激发$\mathbf{k}=0$的自旋波模式,我们可以用相同的方法来讨论。从式\eqref{eq:96}我们可以看出对于一个平行外场,共振频率由下式给出
\begin{equation} \label{eq:99}
\left\{
\begin{array}{l}
\hslash\omega_1=g\beta[H_A(H_A+2H_E)]^\frac{1}{2}+g\beta H;\\
\hslash\omega_2=g\beta[H_A(H_A+2H_E)]^\frac{1}{2}-g\beta H;
\end{array}
\right.
\end{equation}
展示出各向异性在共振现象中扮演着十分重要的角色。垂直情况下的频率可以用相似的方法推导。更多反铁磁共振的自旋波理论可以在参考文献中找到。公式\eqref{eq:99}中的反铁磁共振频率最早由基特尔(Kittel)给出。



%%===========================================================
%%===========================================================

\section{自旋波近似下的反铁磁基态} \label{sec:8}

我们现在将讨论第\ref{sec:7}中定义的二子晶格模型下的反铁磁基态。这是式\eqref{eq:79}中哈密顿量的最低态,代表了近邻自旋间的各向同性相互作用和与假定的两个子晶格结构相对应的交错各向异性场$H_A$中的自旋各向异性能。特别的,对于$H_A\rightarrow 0$时式\eqref{eq:79}的基态和$H_A\rightarrow\infty$下的完全有序态的差别不大,完全有序态下子晶格$1$中所有自旋指向$N$,子晶格$2$中的自旋指向$S$。

在自旋波近似中,哈密顿量\eqref{eq:79}被式\eqref{eq:82}和\eqref{eq:83}中给出的坐标和动量$Q_i,~P_i$描述的简谐振子系统代替。在第\ref{sec:7}章中,该耦合谐振子系统的一系列近似简振模由式\eqref{eq:87},\eqref{eq:88}和\eqref{eq:91}中的变换给出。在这些变量的项中,如果我们忽略谐振子系统哈密顿量\eqref{eq:84}中的四次项,哈密顿量就会变成式\eqref{eq:94}中的标准形式。在没有外场时,简正模的频率都是正数,正如\eqref{eq:96}式取$H=0$时那样。谐振子系统基态下的简正模都没有被激发,也就是说所有的$\mathfrak{n}_{1k}$和$\mathfrak{n}_{2k}$都消失了。然而,这并不意味着所有代替单个自旋系统中自旋的,最初的解耦谐振子都必须处于最低态。这对于$H_A\rightarrow\infty$是成立的,对应于完全有序态。然而对于一个有限的$H_A$特别是当$H_A=0$时,存在一个不为零的$p_n$,代表一个给定谐振子在其第$n$激发态的概率,我们现在将给出它的计算。只要当$n\geqslant 1$时$p_n$远小于一,则$n=0,~1,~\cdots,~2S$时的$p_n$近似等于,在最初自旋系统的基态下,在一个给定原子处找到$n$个自旋偏差的概率,因为在$p_n\leqslant 1$的条件下,对于$n\geqslant 1$的谐振子系统的性质近似于自旋系统的性质。当然,在谐振子系统中的$p_n$对所有$n$值都有意义。

为了计算概率$p_n$关于各向异性场$H_A$的函数,引入产生和湮灭算符可以方便计算。对于式\eqref{eq:91}中的自旋波模式,这些算符由下式定义
\begin{equation} \label{eq:100}
A_{1k}=2^{-\frac{1}{2}}(Q_{1k}'+iP_{1-k}');~~A_{1k}^*=2^{-\frac{1}{2}}(Q_{1-k}'-iP_{1k}'),
\end{equation}
对于模式$2k$也是类似的。对应于原子$i$上的自旋偏差数的自旋偏差算符$n_i$,通过式\eqref{eq:87},\eqref{eq:88}和\eqref{eq:91}中的变换可以用\eqref{eq:100}式中的算符来表示。其结果为
\begin{equation} \label{eq:101}
n_i=(2/N)\sum\nolimits_{k,k'}\exp[i(\mathbf{k}'-\mathbf{k})\cdot\mathbf{r}_i]B_{k'}^*B_k,
\end{equation}
其中
\begin{equation} \label{eq:102}
B_k=c_{1k}A_{1k}+c_{2k}A_{2k}^*.
\end{equation}
此时基态$\psi_0$由下式表征
\[
A_{1k}\psi_0=0,~~A_{2k}\psi_0=0,
\]
基态$\psi_0$下$n_i$的期望值并不依赖与$i$并等于
\begin{equation} \label{eq:103}
\braket{n_i}=(2/N)\sum\nolimits_i\braket{n_i}=(2/N)\sum\nolimits_k\braket{B_k^*B^*},
\end{equation}
其中求和遍及$i$所属子晶格中的所有原子。$B_k^*B_k$中只有一项会对期望值产生不为零的贡献,即$A_{2k}A_{2k}^*$,所以我们有
\begin{equation} \label{eq:104}
\braket{n_i}=(2/N)\sum\nolimits_kc_{2k}^2=\Gamma(1+H_A/H_E),
\end{equation}
其中$\Gamma(x)$函数由下式定义
\begin{equation} \label{eq:105}
\Gamma(x)=(2/N)^\frac{1}{2}\sum\nolimits_k\tfrac{1}{2}[(1-\gamma_k^2/x)^{-\frac{1}{2}}-1],
\end{equation}
而$\gamma_k$由\eqref{eq:90}式给出。

相似地,我们可以计算$n^i$高次项的期望值。比如$n_i^2$我们可以得到
\begin{eqnarray} \label{eq:106}
&&\braket{n_i^2}=(2/N)^2\sum\nolimits_{k,k'}\sum\nolimits_{l,l'}\delta(k+l-k'-l')\nonumber\\
&&~~~~~~~~~~~~~~~~~~~~~~~~~~~~~~~~~~~~~~~~~~~\times\braket{B_{k'}^*B_kB_{l'}^*B_l}.
\end{eqnarray}
此致有两个不为零的项,所以结果为
\begin{equation} \label{eq:107}
\braket{n_i^2}=\Gamma+2\Gamma^2,
\end{equation}
其中用到了关系式$c_{1k}^2=1+c_{2k}^2$。最后,我们计算了$\braket{n_i^3}$。此时存在六个不为零的项,即
\begin{equation} \label{eq:108}
\braket{n_i^3}=\Gamma+6\Gamma^2+5\Gamma^3.
\end{equation}
期望值$\braket{n_i^m}$和概率值$p_n$的关系由下面的公式给出
\begin{equation} \label{eq:109}
\braket{n_i^m}=p_1+2^mp_2+3^mp_3+\cdots,
\end{equation}
因此我们得到了下面一系列关于$p_n$的方程:
\begin{equation} \label{eq:110}
\left\{
\begin{array}{l}
p_0+p_1+p_2+p_3+\cdots=1;\\
~~~~p_1+2p_2+3p_3+\cdots=\Gamma;\\
~~~~p_1+4p_2+9p_3+\cdots=\Gamma+2\Gamma^2;\\
~~~p_1+8p_2+27p_3+\cdots=\Gamma+6\Gamma^2+5\Gamma^3.\\
~~~~~~~~~~~~~~~\cdots
\end{array}
\right.
\end{equation}
通过将$\Gamma$视为小量进一步近似,我们可以解除该方程组。只考虑到$\Gamma$的三次项,我们可以得到
\begin{equation} \label{eq:111}
\left\{
\begin{array}{l}
p_0=1-\Gamma+\Gamma^2-(5/6)\Gamma^3;\\
p_1=\Gamma-2\Gamma^2+(5/2)\Gamma^3;\\
p_2=\Gamma^2-(5/2)\Gamma^3;\\
p_3=(5/6)\Gamma^3.
\end{array}
\right.
\end{equation}
从式\eqref{eq:105}我们可以看出当$H_A\rightarrow\infty$时我们有$\Gamma\rightarrow 0$,因此$p_0=1,~p_1=p_2=\cdots=0$,证明了当各向异性很强时,基态接近于完全有序态。对于$H_A\rightarrow 0$,$\Gamma$趋于一个有限值,安德森(Anderson)用积分项$J_D$来表示:
\begin{equation*}
\Gamma(1)=\tfrac{1}{2}(J_D-1),
\end{equation*}
其中$D$是晶格维度。对于一个 3 维简单立方格子我们可以得到$\Gamma(1)=0.078$,对应于下面的$p_n$值:
\begin{equation} \label{eq:112}
p_0=0.93;~~p_1=0.06;~~p_2=0.01.
\end{equation}
从这些值可以看出,$H_A\rightarrow 0$时的基态和完全有序态差别并不大,这一结果证明了对反铁磁基态使用自旋波近似的合理性,甚至在近似最差的$S=\tfrac{1}{2}$情况下也是合理的。

式\eqref{eq:112}中给出的$p_n$值是与自旋量子数$S$无关的,因为量$\Gamma(1)$仅依赖于晶格结构。因为只有最开始几个$p_n$值明显不等于$0$,所以在经典极限$S\rightarrow\infty$下,自旋是完全排成一列的甚至在没有各向异性的情况下。当然也应该如此,因为众所周知在经典自旋算符的表示下,完全有序态是能量最低的态。对于一个有限值$S$,基态能量可以用式\eqref{eq:96}和\eqref{eq:97}计算。更对细节请参看安德森(Anderson)和库博(Kubo)的论文。

最后,自旋波理论与反铁磁性有关的重大作用是,对于各种作者所使用的半现象学模型(其中一个子晶格上的自旋指向$N$,另一子晶格上的自旋指向$S$),真的存在一定的基础。式\eqref{eq:112}中的结果展示,也就是说这个模型的只有大概$7\%$的误差。沙尔(Shull)和斯马特(Smart)的中子衍射实验明确证明了联锁子晶格模型在物理上是真是存在的。尽管如此,它保证了被理论预言的结果是相同的。令人担心的是,尽管两个子晶格的总旋转都是非对角的,但它们的总和是对角的,但联锁结构可能会被非对角的微扰完全破坏。也可以提出这样的反对意见,即没有办法确定哪个子晶格点$N$和$S$,而不是相反。实际上,存在一个量子力学共振效应,因此两个排列彼此互换。然而对于任何宏观尺寸的晶体来说,这几乎是一个地质学上的缓慢性过程,因为共振仅是由于微扰理论运用于晶格中原子数$N$的结果。因此,对于所有意图和目的,像重分子的异构形式一样,将一种自旋类型分配给一个子晶格,将一种自旋类型分配给另一子晶格的安排可以认为是稳定的。



%%===========================================================
%%===========================================================

\section{亚铁磁简谐近似} \label{sec:9}

过去十年磁学的杰出发展之一是亚铁磁性概念的发展和成果。亚铁磁介质的概念是奈尔(N\'eel)提出的;它与反铁磁材料的不同之处在于,两个子晶格的磁矩是不相等的,这是自旋,$g$因子或格点数不相等的结果,因此即使两个子晶格排列反平行,晶体的合成磁矩也不为零。前几章的自旋波理论可以很容易地被应用到亚铁磁介质上,其中两个子晶格有相同的原子数,而且唯一重要的耦合存在于位于不同子晶格的近邻原子间。真实的亚铁磁介质通常包括更加复杂的几何结构和除了最近邻原子间的原子相互作用。存在同一子晶格自旋耦合的情况下的自旋波理论由卡普兰(Kaplan)给出。然而通过简化模型,只需小小修改一下前几章中的计算结果。

为了对比真正的反铁磁情况,我们不能再假设两个子晶格有相同的$g$因子,自旋和各向异性场。作为方程\eqref{eq:80}和\eqref{eq:81}的代替,我们现在有
\begin{eqnarray} \label{eq:113}
&&\mathcal{H}=2J\sum\nolimits_{\mathit{nei}}\mathbf{S}_i\cdot\mathbf{S}_j-(H+H_{A1})g_1\beta\sum\nolimits_iS_{iz}\nonumber\\
&&~~~~~~~~~~~~~~~~~~~~~~~~~~~~~~~~~~~~~~~~~-(H-H_{A2})g_2\beta\sum\nolimits_jS_{jz}
\end{eqnarray}
和
\begin{equation} \label{eq:114}
n_i=S_1-m_i,~~n_j=S_2+m_j.
\end{equation} 
我们分别用$S_1$和$S_2$代替式\eqref{eq:82}和\eqref{eq:83}中的$S$,并引入合适的简谐振子变量去代替自旋变量。不同于式\eqref{eq:84},现在谐振子系统的哈密顿函数是
\begin{eqnarray} \label{eq:115}
&&\mathcal{H}=E_0''+2J(S_1S_2)^\frac{1}{2}\sum\nolimits_{\mathit{nei}}(Q_iQ_j-P_iP_j)-2J\sum\nolimits_\mathit{nei}n_in_j\nonumber\\
&&~~~~~~~~~~~~~~~~~+[2JS_2z+g_1\beta(H+H_{A1})]\sum\nolimits_in_i\nonumber\\
&&~~~~~~~~~~~~~~~~~~~~~~+[2JS_1z+g_2\beta(H-H_{A2})]\sum\nolimits_jn_j,
\end{eqnarray}
其中
\begin{eqnarray} \label{eq:116}
&&E_0''=-NzJS_1S_2-N(g_1S_1-g_2S_2)\beta H+\nonumber\\
&&~~~~~~~~~~~~~~~~~~~~~~~~~~~~-Ng_1S_1\beta H_{A1}-Ng_2S_2\beta H_{A2}.
\end{eqnarray}
表达式\eqref{eq:115}和\eqref{eq:84}有着完全一样的结构;唯一的不同是系数不一样,而常亮\eqref{eq:116}现在包含一个和外场成正比的项。因此通过和我们在第\ref{sec:7}章使用过的基本一样的变换,哈密顿函数\eqref{eq:115}可以被化简为平方和的形式。当考虑到常数的新值,简正模的频率表达式变成
\begin{eqnarray} \label{eq:117}
&&\hslash\omega_{1k},\hslash\omega_{2k}\nonumber\\
&&~~~=\{[J(S_1+S_2)z+(K_1+K_2)+\tfrac{1}{2}(g_1-g_2)\beta H]^2\nonumber\\
&&~~~~~~~-4J^2z^2S_1S_2\gamma_k^2\}^\frac{1}{2}\pm[j(S_2-S_1)z\nonumber\\
&&~~~~~~~~~~~~~~~~~~~~~~~~~~~~~~+\tfrac{1}{2}(g_1+g_2)\beta H+(K_1-K_2)],
\end{eqnarray}
而非式\eqref{eq:96},其中$+$号对应股$\omega_{1k}$而$-$号对应于$\omega_{2k}$。这里$\gamma_k$的定义由\eqref{eq:90}式给出。根据推导出式\eqref{eq:117}的库维尔(Kouvel)和布鲁克斯(Brooks),我们使用缩写
\begin{equation} \label{eq:118}
K_1=\tfrac{1}{2}g_1\beta H_{A1};~~K_2=\tfrac{1}{2}g_2\beta H_{A2}.
\end{equation}
对于本征值而言,公式\eqref{eq:95}中的附加常数的表达式\eqref{eq:97}现在取$E_0''$而非$E_0'$。表达式$K_1$和$K_2$可以被视作两种子晶格的等效各向异性常数。[公式\eqref{eq:82}给出形如$K_iS_i\times\sin^2(\mathbf{S}_i,z)=K_iS_i^{-1}(S_{ix}^2+S_{iy}^2)$的等效各向异性能,正如前面的修改那样,除了一个附加常数外便与$K_i\tfrac{1}{2}(P_i^2+Q_i^2-1)=K_in_i$一样,所以可以用式\eqref{eq:115}中的$H_{Ai}$项修正。]

亚铁磁共振吸收频率的公式对应于完全对称模式$\mathbf{k}=0$时的频率,因此可以通过式\eqref{eq:117}取$\gamma_k=1$得到。这样得到的公式比最初由基特尔(Kittel)给出的更加普遍,因为他假设了两个子晶格有相等的交换作用场和各向异性场。然而我们的计算还是太特例了,因为我们假设了外场平行于各向异性场。对于任意方向,这种假设是不真实的,而且我们理论中最有趣的部分就是对方向的依赖性。更加普遍的情况被望休斯(Wangsness)等人考虑过。共振频率直接从运动方程推导出,但是无疑地,自旋波理论也可以做到。望休斯(Wangsness)表明,如果分子场系数和净磁化强度的乘积远大于施加的外场和各向异性场,亚铁磁材料共振频率的表达式就可以简化为普通铁磁介质的表达式。

如果所施加外场和各向异性场的影响相对来说不是那么重要,亚铁磁材料的比热和磁矩偏差在低温下应该和$T^\frac{3}{2}$成正比,正如在铁磁材料的情况中一样。这个结果第一眼看去可能令人吃惊,因为根号的出现可推测地复杂化了分析。然而重要的是,低温时只有低特征频率的模式被明显激发,因此只有一个分支,即式\eqref{eq:117}中对应于正号的$1k$支在低温下发挥重要作用,它在长波极限下当$H,~K_1$和$K_2$等于零时有一个不为零的频率。如果我们取$K_1=K_2=0$,而且如果$1-\gamma_k^2\ll 1$,式\eqref{eq:117}中的低频由下式给出
\begin{equation} \label{eq:119}
\hslash\omega_{1k}=\frac{g_1S_1-g_2S_2}{S_1-S_2}\beta H+\frac{2JS_1S_2}{S_1-S_2}\sum\nolimits_aa^2k^2\cos\theta_{k,a}.
\end{equation}
这一表达式和对应铁磁情况的表达式\eqref{eq:22}仅差了一个比例常数。因此作为式\eqref{eq:28}和\eqref{eq:31}的代替,我们立即可以得到
\begin{equation} \label{eq:120}
M_z=M_{z0}\left[1-0.1187\left\{\frac{kT(S_1-S_2)}{2JS_1S_2}\right\}^\frac{3}{2}\right],
\end{equation} 
和
\begin{equation} \label{eq:121}
C_V=cNk\left(\frac{S_1-S_2}{4JS_1S_2}\right)(kT)^\frac{3}{2},
\end{equation}
其中$c$和式\eqref{eq:31}中一样。结果\eqref{eq:121}来自式\eqref{eq:31},因为比较式\eqref{eq:119}和\eqref{eq:22}可以看出,相比于铁磁情况$J$可以由一个因子$[S_1S_2/S(S_1-S_2)]$有效地修正。为了从式\eqref{eq:28}得到式\eqref{eq:120},我们额外利用了每单位谐振子量子数的磁矩偏差被一个因子$g^{-1}(g_1S_1-g_2S_2)/(S_1-S_2)$修正的事实,正如我们通过比较式\eqref{eq:119}和\eqref{eq:22}中$H$系数看到的那样,而且现在$M_{z0}$等于$(g_1S_1-g_2S_2)\beta$而不是$g\beta$。

结果\eqref{eq:120}和\eqref{eq:121}第一次是由库维尔(Kouvel)和布鲁克斯(Brooks)通过半经典分析而非自旋波方法得到的。正如我们期望的那样,第\ref{sec:4}章中的标准铁磁公式对应于$S_1=-S_2=S$的特殊情况。

\textit{反铁磁体的比热。}——正如我们在第\ref{sec:7}章研究的那样,反铁磁介质有$S_1=S_2$。这一情况需要特殊处理。在没有施加外场时,磁化强度自然为零,式\eqref{eq:119}就不能再用了因为分母为零。相反,对于$H=0$和$K_1=K_2=0$的情况,此时两个分支$1k$和$2k$简并,而对于没有施加外场和各向异性常数为零的情况,此时低频由下式给出
\begin{equation} \label{eq:122}
\hslash\omega_k=[4J^2S^2z^2(1-\gamma_k^2)]^\frac{1}{2}=2JSka[z\sum\nolimits_a\cos^2\theta_{k,a}]^\frac{1}{2}.
\end{equation}
对于一个简单或者体心立方晶格[参看公式\eqref{eq:22}后的讨论],这一关系式变成
\begin{equation} \label{eq:123}
\hslash\omega_k=2JSkl(2z)^\frac{1}{2},
\end{equation}
其中$z$是最近邻原子数。因此内能的表达式不再是\eqref{eq:29}时,而是
\begin{eqnarray} \label{eq:124}
&&U=2(L/2\pi)^34\pi\int_0^\infty\frac{[2JSk_0l(2z)^\frac{1}{2}]k_0^2\mathrm{d}k}{\exp(2JSk_0l(2z)^\frac{1}{2}/kT)-1}\nonumber\\
&&~~=2(L/l)^3\frac{(kT)^4}{2\pi^2[2JS(2z)^\frac{1}{2}]^3}\\
&&~~~~~~~~~~~~~~~~~~~~~~~~~\times\int_0^\infty V^3(e^{-V}+e^{-2V}+\cdots)\mathrm{d}V.\nonumber
\end{eqnarray}
最前面的因子$2$是当$H=0$时由式\eqref{eq:96}给出的频率双重简并的统计权重。对应于式\eqref{eq:124}的比热$C_V=(\mathrm{d}U/\mathrm{d}T)$为
\begin{equation} \label{eq:125}
C_V=2\left(\frac{L}{l}\right)^3\frac{4!k}{2\pi^2}\left(\frac{kT}{2JS(2z)^\frac{1}{2}}\right)^3\zeta(4).
\end{equation}
对于一个简单和体心立方晶格$(N/l)^3$的值分别为$N$和$\tfrac{1}{2}N$,而$z$的值分别为$6$和$8$。简单立方格子情况下式\eqref{eq:125}的数值为
\begin{equation} \label{eq:126}
C_V=13.7Nk(kT/12JS)^3,
\end{equation}
这也是库维尔(Kouvel)和布鲁克斯(Brooks)得到的结果之一。

\textit{与实验进行比较。}——所预测的亚铁磁和反铁磁材料的比热是令人震惊的不同,分别与$T^\frac{3}{2}$和$T^3$成正比。库维尔(Kouvel)在实验上证明了磁铁矿的$T^\frac{3}{2}$依赖性。他对亚铁磁非导体的测量提供了可能对整个自旋波理论来说最直接的实验证据。当然,有很多磁性共振的实验,但是共振频率只包含十分特别的对称态的性质。验证比热理论比验证磁化强度随温度变化的理论更简单,因为磁化强度对其最大值的偏离在$T=0$附近是很小,因此自旋波理论可以适用。更深入地,比起普通铁磁材料,对铁氧体的测量是更加确切的,因为普通铁磁材料是导体所以导带结构是简单的理论复杂化,正如我吗在第\ref{sec:11}章中强调的那样。相比于我们所得到的结果,库维尔(Kouvel)给出的磁铁矿理论包含一个不同的$T^\frac{3}{2}$定律比例常数,因为这种材料是尖晶石型结构而不是简单立方结构。库维尔(Kouvel)有趣的论文给出了相关细节和关于从实验上测试自旋波理论的其他结论的可能性的讨论。和晶格振动导致的比热相比,反铁磁的$T^3$公式包含与之相同的温度依赖形式。然而,库维尔(Kouvel)发现磁性贡献可能包含一个比正常德拜晶体项大更多的比例常数,因此应该可以从实验上检测到。但是,各向异性效应的修正破坏了$T^3$定律的严格性。



%%===========================================================
%%===========================================================

\section{自旋-弛豫过程的简谐振子模型} \label{sec:10}

随着微波光谱的发展,弛豫过程引起了科学家们越来越多的兴趣,通过弛豫过程,各种自旋状态彼此之间以及与晶格振动的温度达到热平衡。两种特征弛豫时间需要被区分,即自旋-自旋弛豫时间和自旋-晶格弛豫时间。前者通常更短,而且是由破坏了磁矩空间分量不变形的偶极或者赝偶极相互作用导致的。后者是由交换作用能或者偶极能被晶格振动调制而导致的。我们不会详细讨论弛豫过程的理论;他们十分复杂,关于弛豫过程是否大到产生重要影响,有时又是自相矛盾的。然而,我们确实想概述如何根据我们的谐振子模型来拟定和直观显示各种机制。弛豫相关项在每种情况下都是非谐微扰,它会引起原本不受干扰的谐振子状态间的跃迁。

首先让我们来考虑两者间较简单的自旋-自旋弛豫。由偶极或赝偶极势的$S_{ix}S_{jz}$或$S_{iy}S_{jy}$部分引起的式\eqref{eq:39}中的在$S_z'$中非对角的非谐项,将体现无微扰自旋波的不同谐振子耦合在一起。因此,从理论上讲,研究确保自旋波之间热平衡过程的问题与研究非谐项如何干扰晶格振动并产生有限的热导率的问题相似。计算得到的高温下自旋-自旋弛豫时间的数量级是合理的,但是基于偶极或赝偶极相互作用的普通模型导致$T=0$时磁性共振线谱的宽度为零,或几乎为零,这与实验相反。如果所有磁性原子都位于相似的位置,则唯一对线谱宽度有贡献的相互作用就是那些自旋波的能量和动量守恒的相互作用。因此,在完全均质的材料中,磁性共振线谱的加宽必须由更高能态的微扰引起,因为基态能量低于任何其他状态。因此更高能级只能被三次非谐项加宽,其中一个对称的,或者基特尔(Kittel)量子\eqref{eq:69}退激发,并且两个半基特尔频率但是波矢大小相等方向方向的自旋波被激发。这一过程由凯弗(Keffer)以及基特尔(Kittel)和亚伯拉罕斯(Abrahams)给出。在我们看来,几乎是不可能在低温下足以给出明显的加宽。布隆伯根(Bloembergen)和王(Wang),以及粕谷(Kasuya)也得到了相同的结论,粕谷(Kasuya)还研究了更高次非谐过程,但发现它们在$T=0$时几乎是无效的。

长期以来,低温下铁磁共振线谱有限宽度的来源是一个令人困惑的问题。然而这一神秘面纱最近已经在铁氧体材料情况下被揭开。在如$\mathrm{NiFe_3O_4}$的材料中,$\mathrm{Ni^{\mathit{++}}}$和$\mathrm{Fe^{\mathit{+++}}}$离子在各种八面体格点之间的不规则分布提供了必要的不均匀性。即使这样还是有一个难点。直到最近,我们才认为在接近对称的或者基特尔(Kittel)波$\mathbf{k}=0$附近没有其他频率的自旋波。如果是这种情况,则由不规则引起的微扰不足以提供足够大的线宽。然而,克拉斯顿(Clogston),苏尔(Suhl),沃克(Walker)和安德森(Anderson)发现,在有限尺寸的样品中,低波数的自旋波频率很大程度上收到边界条件影响,因此可以获得在$k=0$附近的简并性,这是谱线加宽的前提。宽度应该取决于晶体如何被切割,而结果正是这样。

在克拉斯顿(Clogston)等人之前的文章里,高尔特(Galt),耶格尔(Yager)和梅利特(Merritt)提出了决定铁氧体材料线宽的另一种机制,即八面体格点同时包含亚铁和铁离子。比如$\mathrm{Fe_3O_4}$或者其他包含额外离子杂志的铁氧体。二价和三价格点将通过电子转移重新连续地分布。因此为衰减和谱线加宽提供了一个动态机制,在转移频率和信号频率相当时特别有效。

在前两段中描述的铁氧体线谱增宽机制并不能应用到普通金属比如镍。导电金属的不均匀性的可能来源及其导致的在$T=0$时的有限线宽,可能是由于原子在离子化不同阶段($d^8,d^9,等$)中的不规则分布,甚至是晶格缺陷引起的。但是只有在不均匀性的波动没有快到无效的情况下,才能确保必要的线宽;电子传导可以很好地使不同价态分配给不同晶格位点。另一种完全不同的铁磁共振线宽机理由阿门特(Ament)和拉多(Rado)提出,而且麦克唐纳(MacDonald)之前的分析也给出了暗示。这些作者说明了趋肤效应式信号波衰减并且加宽了共振线宽。

共振谱线的宽度表明自旋-自旋相互作用还需要更加深入的理论和实验研究。格里森(Gerritsen)对反铁磁共振的实验表明对于反铁磁而言,谱线宽度在$T=0$时区域零。

石榴石和铁氧体不同之处在于,石榴石中不可能发生不同价态离子在不同晶格格点上的重新分布,我们确实也发现了石榴石只有十分宅的共振谱线。

让我们现在来考虑自旋-晶格耦合。自旋-晶格弛豫是由自旋相互作用能的调制导致的,即晶格振动导致的自旋波。我们可以将出现在我们很多公式,如式\eqref{eq:1},\eqref{eq:2}和\eqref{eq:3}中的交换积分$J_{ij}$和偶极常数$D_{ij}$在代表晶格畸变的正交坐标下展开为幂级数,并且只保留线性项。因为在做此展开前,最初的哈密顿量在自旋波谐振子坐标和动量表示下是二次的,因此我们可以得到如下微扰哈密顿量的显示
\begin{equation} \label{eq:127}
\sum\nolimits_{\mathit{ijk}}(a_\mathit{ijk}Q_iQ_jq_k+b_\mathit{ijk}P_iP_jq_k),
\end{equation}
其中$q_k$是声子,而$Q_i$和$P_i$是自旋波谐振子坐标。因为自旋波和声子都可以被解释为简谐振子,我们又有了谐振子系统的三次非谐性微扰耦合的情况。和自旋-自旋相互作用中不同的是,自旋-晶格相互作用中的一种量子是声子另外两种是自旋波,而自旋-自旋相互作用中三种全都是自旋波。(除了第三阶,可能会有更高阶过程,但是为了简便我们假设三次效应是最重要的。)结构\eqref{eq:127}中的项可以完全来自纯各向同性交换耦合作用,然而在自旋-自旋相互作用中,重要的微扰必定是那些在磁化过程中非对角线的扰动。

在一篇重要论文中,阿希耶尔(Akhieser)十分详细的检验了自旋-自旋和自旋-晶格弛豫效应。然而他的计算结果被波尔德(Polder)批评了,波尔德(Polder)认为经过适当修改,它们不会产生足够短的弛豫时间来与实验相符。阿希耶尔(Akhieser)使用了纯经典偶极相互作用,而基特尔(Kittel)和亚伯拉罕斯(Abrahams)指出更大的扰动是可能存在的,因为自旋-轨道相互作用导致的赝偶极耦合。他们有一个巧妙的方法,即从经验磁弹性常数中提取常数。他们声称以此方式获得了足够大的自旋-晶格相互作用。阿希耶尔(Akhieser)用积分代替求和的方式可能低估了短程交换积分的调制效果,其与距离之间成指数关系。



%%===========================================================
%%===========================================================

\section{海特勒-伦敦(Heitler-London)模型以外的自旋波理论的有效性问题} \label{sec:11}


\begin{eqnarray} \label{eq:128}
&&-2\sum\nolimits_{j>i}J_{ij}\mathbf{S}_i\cdot\mathbf{S}_j=-2J\sum\nolimits_{j>i}\mathbf{S}_i\cdot\mathbf{S}_j\nonumber\\
&&~~~~~~~~~~~~~~~~~~~~~~~~=-J[(\sum\nolimits_i\mathbf{S}_i)^2-\sum\nolimits_i(\mathbf{S}_i)^2]\nonumber\\
&&~~~~~~~~~~~~~~~~~~~~~~~~~~~=-J[S'(S'+1)-NS(S+1)].
\end{eqnarray}

\begin{equation*} 
E=-2S\sum\nolimits_\mathit{ij}J_\mathit{ij}\exp(i\mathbf{k}\cdot\mathbf{R}_\mathit{ij})=-2JNS\sum\nolimits_j\exp(i\mathbf{k}\cdot\mathbf{R}_\mathit{ij}).
\end{equation*}

\begin{equation} \label{eq:129}
\left\{
\begin{array}{l}
E=-2JNS~~~\text{对于}~k=0;\\
E=0~~~~~~~~~~~~\text{对于}~k\neq 0.
\end{array}
\right.
\end{equation}

\section*{致谢}
其中一位作者范克兰东克(JVK)希望感谢荷兰纯科研组织(ZWO)的资助,这使得他得以在1953年留在哈佛大学来写作这篇文章的部分内容。

\end{document}







